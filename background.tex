\chapter{Background}
    \section{Pupillometry}\label{sec:b_p}
	Pupillometry refers to the long-standing physiological method of measuring pupil diameter.
	Sporadic studies involving pupil diameter measurements have been performed as early as the beginning of the XIX. century \cite{Loewenfeld1958},
	with the establishment of a scientific community around pupillometry following about oone and a half centuries later.
	The early emergence of pupillometric research owes to the innovation of infra-red imaging to study the eye \cite{Dubois1955}, around the middle of the XX. century (originally for ophthalmologic purposes, where pupillometry also continues to be used \cite{Thompson2012}).
	This development made it possible to measure pupil kinetics without interference from the measurement equipment itself - and remains a standard for even the newest devices of our day \cite{Bradley2010}.
	
	Pupillometry has been in documented use in psychology since 1960, when Hess and Polt published what was arguably the seminal paper of psychophysiological pupillometry in the journal \textit{Science} \cite{HESS1960}.
	In this publication they reported presenting images of partly nude adults and babies to a small sample of participants, and observing increased pupil dilation in males to stimuli depicting women, and increased pupil dilation in women to stimuli depicting either males or babies.
	In light of this manifold interpretable finding, research over the next decade has focused on making the case that pupil dilation is a marker of attraction, or well-being.
	
	Many early publications present varying concepts related to the initial publication by Hess and Polt, and report a robust modulation of pupil diameter by sexual attraction \cite{Goldwater1972, HESS1965} or positive affect \cite{Nunally1967, Bradshaw1967}.
	Even some of these early papers, however, hinted at a broader phenomenon underlying neuropsychological control of pupil size.
	This broader phenomenon was initially termed “general activation”\cite{Nunally1967}, and has over the years been more accurately identified as “cognitive load”\cite{Zekveld2011}, “attention” \cite{Wykowska2013,Kraemer2000}, or - most recently - “control state” \cite{Hayes2013}.
	Recently, researchers have tentatively tried to establish pupillometry as an indicator of neuronal activity in the Locus Coeruleus (more on this in section~\ref{sec:b_lc}) \cite{Gilzenrat2010}.
	In the research herein presented, we have sought to test the viability of pupillometry as an assessment not of a neuropsychological phenomenon, but of a neurological phenomenon - 
	complementing previous approaches\cite{Gilzenrat2010} (mostly behavioural \cite{Granholm2004}) with concomitant brain imaging.
	
    \section{Pupil Dilation Response}
	Pupilary dilation (mydriasis) is most prominently known as a physiological reaction to low light conditions \cite{Ellis1981}. 
	However, a broad range of mental states - 
	such as alertness\cite{Yoss1970}, arousal\cite{Bradshaw1967}, fatigue\cite{Morad2000}, and surprize\cite{Preuschoff2011} -
	have also been shown to modulate pupil dilation.
	Additionally, application of numerous substances (topically and/or systemically) is known to affect pupil dilation \cite{Theofilopoulos1988, Bye1979, Phillips2000}.
	
	The direct control mechanism of pupil dilation (or contraction for that matter) are two sets of smooth muscles located in the iris: the iris dilator muscle and the iris sphincter muscle.
	\begin{itemize}
	    \item The iris dilator is a radial muscle located anterior of the pigment epithelium of the iris, and chiefly responsible for iris dilation.
	    This muscle is innervated by the sympathetic nervous system an is controlled via noradrenergic stimulation to its $\alpha_1$-adrenergic receptors \cite{vanAlphen1976}.
	    \item The iris sphincter muscle is located anterior of the iris dilator muscle and chiefly responsible for iris contraction.
	    Its innervation stems from the parasympathetic system and is mainly mediated by M$\mathsf{_3}$-muscarinic acetylcholine receptors \cite{Woldemussie1993,Taylor1974}.
	\end{itemize}
	It should be noted that counterbalanced neurotransmitter effects (such as direct cholinergic inhibition of the iris dilator muscle) have also been documented \cite{Yoshitomi1985}.
	
	In light of these characteristics it is not surprising that topical application of drugs can affect pupil dilation.
	This phenomenon is especially robust and well-documented for cholinergic antagonists \cite{Gambill1967} -
	to the extent that pupil dilation is used as an assay for in vivo anticholinergic effects \cite{Bye1979}.
	Serotonin (5-HT) receptors have also been shown to affect pupil dilation - both by topically applied serotonin \cite{KOELLA1962}, or by systemically applied receptor agonists \cite{Yu2004}.
	Systemic application of 5-HT agonists (in rats) has furthermore been shown to elicit mydriasis not via peripheral 5-HT receptors, but via CNS 5HT$\mathsf{_{1A}}$ receptors.
	The hypothesis that systemic monoamine receptor antagonist application would modulate pupil dilation via the aforementioned robust peripheral mechanisms has also been tested on the example of $\alpha_1$-adrenoceptor antagonists in humans \cite{Cooney2012}.
	We would like to note that this study has also failed to find a peripherally mediated efect upon systemic application. 
	
	Considerable attention was received by the hypothesis that CNS 5-HT (in liaison with depression, and a long-standing interest of pupillometrists in affect - detailed under section~\ref{sec:b_p}) may be chiefly implicated in controlling pupil dilation.
	As a most prominent example of pupil-dilation and serotonin interaction, antidepressants of the selective serotonin re-uptake inhibitor (SSRI) class are widely documented to cause tonic mydriasis \cite{Fitzgerald2013,Klein-Schwartz2012}.
	
	Relying on this established phenomenon, at least one study has tried to investigate the value of pupil dilation as a convenient, semi-quantitative 5-HT biomarker \cite{Noehr-Jensen2009}.
	The study found an escitalopram-correlated reduction of pupil constriction, but no overall effect of escitalopram (which the authors use as a proxy for systemic 5-HT) on pupil dilation.
	The authors conclude that pupillometry can, thus, not be used as a 5-HT biomarker; though in this context we would like to observe that theirs was a chronic test for tonic dilation.
	In fact, the modulation of pupillary light reflex the authors found, may even indicate that the outcome of the study would have been quite different, had it used pupillometry to assay phasic dilation related to acute events (such as for instance in a behavioural task).
	In any case, the finding of the authors conflicts with the often reported dose-dependent pupil dilator activity of SSRIs \cite{Nielsen2010,Fitzgerald2013,Klein-Schwartz2012};
	and need not dismiss all hopes of pupillometry as a serotonin biomarker.
	
	Evidence has mounted considerably faster and more consistently on establishing pupil diameter as a biomarker for norepinephrine (NE).
	In this context it my be hypothesized that 5-HT works upstream of NE in controlling pupil dilation, and this may lead to the considerably opaquer situation for 5-HT than for NE.
	Research into the pupil dilation effects of 3,4-methylenedioxymethamphetamine (MDMA) administered in the presence of varying monoamine reuptake inhibitors and monoamine agonists has, however, shown that both 5-HT and NE act on pupil dilation independently of each other \cite{Hysek2012}.
	Though this finding cannot completely exclude synergistic effects, we believe it reasonable to investigate NE and 5-HT mediated pupil dilation separately.
	
	
	
    \section{Locus Coeruleus}\label{sec:b_lc}
	\iffalse
	he method of pupillometry has been used for over half a century ( http://apps.webofknowledge.com/full_record.do?product=UA&search_mode=CitingArticles&qid=12&SID=U1PK6FCo9fkA@A8KlBk&page=2&doc=17&cacheurlFromRightClick=no ), and has been popular throughout subfields of psychology, as well as in marketing and economics studies ( http://www.ncbi.nlm.nih.gov/pubmed/15003368 ) - in later years the method has lost in popularity due to its fuzziness (interacting, as detailed above, with many kinds of mental states) and due to poor understanding of the underlying mechanics.
	
	
, and has been popular throughout subfields of psychology, as well as in marketing and economics studies ( http://www.ncbi.nlm.nih.gov/pubmed/15003368 ) - in later years the method has lost in popularity due to its fuzziness (interacting, as detailed above, with many kinds of mental states) and due to poor understanding of the underlying mechanics. It has been repeatedly shown that in higher mammals activation of the 5HT-1A receptor or bathing the iris in serotonin causes miosis ( http://www.sciencedirect.com/science/article/pii/S0014299904002201 , http://www.ncbi.nlm.nih.gov/pubmed/14034114?myncbishare=&otool=idehulib ) - however, serotonergic psychedelics (LSD, psilocybin, mescaline, etc.) which agonize the 5HT-2A receptor commonly cause mydriasis. In light of antidepressants of the SSRI class also exhibiting similar effects ( http://www.ncbi.nlm.nih.gov/pubmed/11919665 ) at least one study has been conducted to investigate the value of pupillometry as a semi-quantitative serotonin biomarker ( http://www.ncbi.nlm.nih.gov/pubmed/19404631 ) the study failed to replicate previous results showing mydriasis upon citalopram intake and concluded that pupillometry is not a reliable biomarker for serotonin. The state of the topic is uncertain, as the initial SSRI-mydriasis results have been repeatedly and independently replicated over the last decades ( http://www.ncbi.nlm.nih.gov/pubmed/2949057 , http://www.ncbi.nlm.nih.gov/pubmed/2361813 ). One proposed source of varying results was that SSRIs influence pupil dilation via increase in sympathetic noradrenaline release - recent studies, however, show that SSRIs actually inhibit the firing rate of noradrenergic neurons in the long term ( http://www.ingentaconnect.com/content/apl/pcp/2007/00000011/a00201s2/art00004 ). These findings indicate that pupil size may synergistically be determined by both serotonin and noradrenaline. In fact, on the noradrenergic front pupillometry has become established as a reliable biomarker for locus ceruleus activation ( http://www.ncbi.nlm.nih.gov/pubmed/20557480 ).

	In light of the above, I find that a coupled fMRI/pupillometry study would be very much enlightening (amongst others, seeing as the locus ceruleus connection has only been shown via EEG). Additionally, if it is possible to collect blood/urine samples to measure serotonin and/or use an appropriate measure for norepinephrine (?), that would further establish or dismiss pupillometry as a reliable method for modern research. In any case (even if all serotonin/noradrenalin) correlations should fail, we can be confident in obtaining pupil dilation results if we do an attractive faces trial: The original research which sparked pupillometry was based on trials with either attractive faces or naked bodies ( Hess & Polt, 1960, 1964; Kahneman & Beatty, 1966) .We could  then analyze the amplitude of the dilation together with the fMRI data (to the best of my knowledge for the first time).

	I have taken the kinetics description of the pupil dilation response (PDR) from this paper ( http://www.ncbi.nlm.nih.gov/pubmed/20557480 ) - they analyze it n the context of surprize:
	
	
	
	
	http://apps.webofknowledge.com/full_record.do?product=UA&search_mode=CitingArticles&qid=12&SID=U1PK6FCo9fkA@A8KlBk&page=1&doc=3&cacheurlFromRightClick=no ), cognitive load ( http://graphics.stanford.edu/papers/visual-cognitive-load/ ), alertness ( http://apps.webofknowledge.com/full_record.do?product=UA&search_mode=CitingArticles&qid=7&SID=U1PK6FCo9fkA@A8KlBk&page=1&doc=4&cacheurlFromRightClick=no ), and many others ( http://www.ncbi.nlm.nih.gov/pubmed/15003368 ) seem to modulate pupil dilation. The method of pupillometry has been used for over half a century ( http://apps.webofknowledge.com/full_record.do?product=UA&search_mode=CitingArticles&qid=12&SID=U1PK6FCo9fkA@A8KlBk&page=2&doc=17&cacheurlFromRightClick=no ), and has been popular throughout subfields of psychology, as well as in marketing and economics studies ( http://www.ncbi.nlm.nih.gov/pubmed/15003368 ) - in later years the method has lost in popularity due to its fuzziness (interacting, as detailed above, with many kinds of mental states) and due to poor understanding of the underlying mechanics. It has been repeatedly shown that in higher mammals activation of the 5HT-1A receptor or bathing the iris in serotonin causes miosis ( http://www.sciencedirect.com/science/article/pii/S0014299904002201 , http://www.ncbi.nlm.nih.gov/pubmed/14034114?myncbishare=&otool=idehulib ) - however, serotonergic psychedelics (LSD, psilocybin, mescaline, etc.) which agonize the 5HT-2A receptor commonly cause mydriasis. In light of antidepressants of the SSRI class also exhibiting similar effects ( http://www.ncbi.nlm.nih.gov/pubmed/11919665 ) at least one study has been conducted to investigate the value of pupillometry as a semi-quantitative serotonin biomarker ( http://www.ncbi.nlm.nih.gov/pubmed/19404631 ) the study failed to replicate previous results showing mydriasis upon citalopram intake and concluded that pupillometry is not a reliable biomarker for serotonin. The state of the topic is uncertain, as the initial SSRI-mydriasis results have been repeatedly and independently replicated over the last decades ( http://www.ncbi.nlm.nih.gov/pubmed/2949057 , http://www.ncbi.nlm.nih.gov/pubmed/2361813 ). One proposed source of varying results was that SSRIs influence pupil dilation via increase in sympathetic noradrenaline release - recent studies, however, show that SSRIs actually inhibit the firing rate of noradrenergic neurons in the long term ( http://www.ingentaconnect.com/content/apl/pcp/2007/00000011/a00201s2/art00004 ). These findings indicate that pupil size may synergistically be determined by both serotonin and noradrenaline. In fact, on the noradrenergic front pupillometry has become established as a reliable biomarker for locus ceruleus activation ( http://www.ncbi.nlm.nih.gov/pubmed/20557480 ).

	In light of the above, I find that a coupled fMRI/pupillometry study would be very much enlightening (amongst others, seeing as the locus ceruleus connection has only been shown via EEG). Additionally, if it is possible to collect blood/urine samples to measure serotonin and/or use an appropriate measure for norepinephrine (?), that would further establish or dismiss pupillometry as a reliable method for modern research. In any case (even if all serotonin/noradrenalin) correlations should fail, we can be confident in obtaining pupil dilation results if we do an attractive faces trial: The original research which sparked pupillometry was based on trials with either attractive faces or naked bodies ( Hess & Polt, 1960, 1964; Kahneman & Beatty, 1966) .We could  then analyze the amplitude of the dilation together with the fMRI data (to the best of my knowledge for the first time).

	I have taken the kinetics description of the pupil dilation response (PDR) from this paper ( http://www.ncbi.nlm.nih.gov/pubmed/20557480 ) - they analyze it n the context of surprize:
	\fi
    \section{Open Science}\label{sec:b_os}
	The format of scientific publishing has become a highly debated topic in recent years - as exemplified by \textit{Nature}'s landmark “Peer Review Trial” of 2006 \cite{Nature-debate2006}.
	Much of the discussion in the scientific community concerns open access \cite{VanNoorden2013,Parker2013}, and the more controversial open peer review.
	While these discussions are formally dealing with publishing formats other than the classical “thesis”, we hold the concerns of the authors relevant to all forms of knowledge presentation.
	
	In creating this document we have kept the following considerations in mind, and used all of the below mentioned software.
	\subsection{Reproducible Data}\label{sec:b_rr}
	    Data reproducibility 
	\subsection{Reproducible Analysis}\label{sec:b_ra}
	    In addition to reproducing the raw material of research - the data - transparent research also relies on reproducibility of analysis \cite{Peng2009}.
	    Especially in fields where presented results rely on numerous statistical manipulations (as many sub-fields of modern neuroscience do) this concern is of utmost relevance.
	    	    
	\subsection{Reproducible Documents}\label{sec:b_rd}
	    The final form of a scientific project - the “publication” or the “document” represents the condensed form of knowledge and as such integrates the most work and is the most difficult to reproduce \cite{Schwab2000}.
	    
	    Reproducibility of a document does not refer merely to recompiling static text (for instance from \colorbox{vlg}{\texttt{.tex}} to \colorbox{vlg}{\texttt{.pdf}}), but rather recompiling a dynamic representation of the results.
	    Therefore, a reproducible document is only to be understood as an additional layer on top of reproducible data analysis, and should avoid wherever possible inclusion of static (copy-paste) measures obtained from the data. 
	    
	    A viable solution herefor is making the document \textit{live} and creating figures and numeric representations of the data at compile time.
	    A prerequisite 
	    
	    
	    For piping both \textit{live} python data frames and \textit{live} matplotlib figures to our document, we have found the PythonTeX \cite{Poore2013} package more than adequate.
	    Live data within a reproducible document can be used for additional micro-analysis (such as t-tests, ANOVA, or linear models).
	    Wrappers for statistical functions - as for instance the interstats package \cite{interstats} - are a convenient way of doing so.
	    More complex statistical methods (such as linear models) may require passing tables from the backend (which can be R\cite{R}, for example) to the \LaTeX\ compiler.
	    There are also a wealth of packages providing direct (and wrappable) R-to-\LaTeX\ piping.
	    Two of the most prominent such packages are texref \cite{Leifeld2013} and stargazer \cite{Hlavac2013}.
