\begin{pythontexcustomcode}[begin]{py}
import subprocess
import sys
sys.path.append('/home/chymera/src/interstats/')
pytex.add_dependencies('/home/chymera/src/interstats/interstats.py')

from scipy.stats import ttest_rel, ttest_ind
from interstats import lm, av, tex_nr, p_table


import matplotlib as mpl
mpl.use("pgf")
pgf_with_custom_preamble = {
    "font.family": "serif", # use serif/main font for text elements
    "text.usetex": True,    # use inline math for ticks
    "pgf.rcfonts": False,   # don't setup fonts from rc parameters
    "pgf.preamble": [
         r"\usepackage{units}",         # load additional packages
         r"\usepackage{metalogo}",
         r"\usepackage{unicode-math}",  # unicode math setup
         r"\setmathfont{xits-math.otf}",
         r"\setmainfont{DejaVu Serif}", # serif font via preamble
         ]
}
mpl.rcParams.update(pgf_with_custom_preamble)

from pylab import *

# Set the prefix used for figure labels
fig_label_prefix = 'fig'
# Track figure numbers to create unique auto-generated names
fig_count = 0

def save_fig(name='', legend=False, fig=None, ext='.pgf', fig_width=1, fig_height=1):
    '''
    Save the current figure (or `fig`) to file using `plt.save_fig()`.
    If called with no arguments, automatically generate a unique filename.
    Return the filename.
    '''
    # Get name (without extension) and extension
    if not name:
        global fig_count
        # Need underscores or other delimiters between `input_*` variables
        # to ensure uniqueness
        name = 'auto_fig_{}_{}_{}-{}'.format(pytex.input_family, pytex.input_session, pytex.input_restart, fig_count)
        fig_count += 1
    else:
        if len(name) > 4 and name[:-4] in ['.pgf', '.svg', '.png', '.jpg']:
            name, ext = name.rsplit('.', 1)

    # Get current figure if figure isn't specified
    if not fig:
        fig = gcf()
    fig.set_size_inches(fig_width,fig_height)
    fig.savefig(name + ext)
    fig.clf()
    return name

def latex_environment(name, content='', option=''):
    '''
    Simple helper function to write the `\begin...\end` LaTeX block.
    '''
    return '\\vspace{0.3cm}\n\\begin{%s}%s\n%s\n\\end{%s}' % (name, option, content, name)

def latex_figure(name=None, caption='', label='', width=0.8):
    ''''
    Auto wrap `name` in a LaTeX figure environment.
    Width is a fraction of `\textwidth`.
    '''
    if not name:
        name = save_fig()
    content = '\\centering\n'
    content += '\\makeatletter\\let\\input@path\\Ginput@path\\makeatother\n'
    content += '\\input{%s.pgf}\n' % name
    if not label:
        label = name
    if caption and not caption.rstrip().endswith('.'):
        caption += '.'
    # `\label` needs to be in `\caption` to avoid issues in some cases
    content += "\\caption{%s\\label{%s:%s}}\n" % (caption, fig_label_prefix, label)
    return latex_environment('figure', content, '[htp]')
\end{pythontexcustomcode}
