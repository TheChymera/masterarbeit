%START FIGURES AND DATA 
\begin{pycode}[pe_ss1]
pytex.add_dependencies('/home/chymera/src/faceRT/analysis/RTforCategories.py')
pytex.add_dependencies('/home/chymera/src/faceRT/analysis/data_functions.py')
pytex.add_dependencies('/home/chymera/src/faceRT/analysis/gen.cfg')
sys.path.append('/home/chymera/src/faceRT/analysis/')
import RTforCategories
data_pe_ss1 = RTforCategories.main(experiment='6px-4px-6steps', prepixelation=0, source='server', elinewidth=1, ecolor='0.3', make_tight={"pad": 0})
fig_pe_ss1 = latex_figure(save_fig(fig_width=6.64, fig_height=3), caption='Reaction times for Hariri-style face matching. Scrambled images for simple visual recognition were preprocessed only with a scrambling cluster of the sizes indicated in the graphic (sizes given in pixels). Reaction times for non-response trials were counted as \SI{5}{\second}. The error bars represent the standard deviation.', label='r_pe_ss1')
\end{pycode}
\begin{pycode}
pytex.add_dependencies('/home/chymera/src/faceRT/analysis/RTdistribution.py')
pytex.add_dependencies('/home/chymera/src/faceRT/analysis/data_functions.py')
sys.path.append('/home/chymera/src/faceRT/analysis/')
pytex.add_dependencies('/home/chymera/src/faceRT/analysis/gen.cfg')
import RTdistribution
keep_scrambling=(0, 6, 22)
data_pe_ss2 = RTdistribution.main(experiment='6px-4px-6steps', prepixelation=0, keep_scrambling=keep_scrambling, source='server', make_tight={"pad": 0}, print_title=False)
fig_pe_ss2 = latex_figure(save_fig(fig_width=6.64, fig_height=3), caption='Histogram of reaction times for Hariri-style face matching. Scrambled images for simple visual recognition were preprocessed only with a scrambling cluster. Missed targets have not been counted. For this graphic we have only indexed the '+ ', '.join([str(i) for i in keep_scrambling]) +' pixel scrambling clusters.', label='r_pe_ss2')
\end{pycode}
\begin{pycode}[pe_ss3]
pytex.add_dependencies('/home/chymera/src/faceRT/analysis/signal_detection.py')
pytex.add_dependencies('/home/chymera/src/faceRT/analysis/gen.cfg')
sys.path.append('/home/chymera/src/faceRT/analysis/')
import signal_detection
data_pe_ss3 = signal_detection.main(experiment='6px-4px-6steps', prepixelation=0, source='server', elinewidth=1, ecolor='0.3', make_tight={"pad": 0})
fig_pe_ss3 = latex_figure(save_fig(fig_width=6.64, fig_height=3), caption='Error rates for Hariri-style face matching. Scrambled images for simple visual recognition were preprocessed only with a scrambling cluster of the sizes indicated in the graphic (sizes given in pixels). The error bars represent the standard deviation. Bar-plots with zero-height have been slightly indented for better visibility.', label='r_pe_ss3')
\end{pycode}
\begin{pycode}
pytex.add_dependencies('/home/chymera/src/faceRT/analysis/RTforCategories.py')
pytex.add_dependencies('/home/chymera/src/faceRT/analysis/data_functions.py')
pytex.add_dependencies('/home/chymera/src/faceRT/analysis/gen.cfg')
sys.path.append('/home/chymera/src/faceRT/analysis/')
import RTforCategories
data_pe_cs1 = RTforCategories.main(experiment='11px-4px-5steps', prepixelation=6, source='server', elinewidth=1, ecolor='0.3', make_tight={"pad": 0})
fig_pe_cs1 = latex_figure(save_fig(fig_width=6.64, fig_height=3), caption='Reaction times for Hariri-style face matching. Scrambled images for simple visual recognition were preprocessed with composite cluster and kernel based scrambling, with a constant kernel of \SI{6}{\pixel} and clusters of the sizes indicated in the graphic (sizes given in pixels). Reaction times for non-response trials were counted as \SI{5}{\second}. The error bars represent the standard deviation.', label='r_pe_cs1')
\end{pycode}
\begin{pycode}
pytex.add_dependencies('/home/chymera/src/faceRT/analysis/RTforCategories.py')
pytex.add_dependencies('/home/chymera/src/faceRT/analysis/data_functions.py')
pytex.add_dependencies('/home/chymera/src/faceRT/analysis/gen.cfg')
sys.path.append('/home/chymera/src/faceRT/analysis/')
import RTforCategories
data_pe_cs2 = RTforCategories.main(experiment='6px-4px-5steps', prepixelation=6, source='server', elinewidth=1, ecolor='0.3', make_tight={"pad": 0})
fig_pe_cs2 = latex_figure(save_fig(fig_width=6.64, fig_height=3), caption='Reaction times for Hariri-style face matching. Scrambled images for simple visual recognition were preprocessed with composite cluster and kernel based scrambling, with a constant kernel of \SI{6}{\pixel} and clusters of the sizes indicated in the graphic (sizes given in pixels). Reaction times for non-response trials were counted as \SI{5}{\second}. The error bars represent the standard deviation.', label='r_pe_cs2')
\end{pycode}
\begin{pycode}[pe_cs3]
pytex.add_dependencies('/home/chymera/src/faceRT/analysis/RTforCategories.py')
pytex.add_dependencies('/home/chymera/src/faceRT/analysis/data_functions.py')
pytex.add_dependencies('/home/chymera/src/faceRT/analysis/gen.cfg')
sys.path.append('/home/chymera/src/faceRT/analysis/')
import RTforCategories
data_pe_cs3 = RTforCategories.main(experiment='11px-4px-5steps_narrow-angle', prepixelation=6, source='server', make_std=True, elinewidth=1, ecolor='0.3', make_tight={"pad": 0})
fig_pe_cs3 = latex_figure(save_fig(fig_width=6.64, fig_height=3), caption='Reaction times for Hariri-style face matching. Scrambled images for simple visual recognition were preprocessed with composite cluster and kernel based scrambling, with a constant kernel of \SI{6}{\pixel} and clusters of the sizes indicated in the graphic (sizes given in pixels). Reaction times for non-response trials were counted as \SI{5}{\second}. The light error bars represent the standard deviation, while the dark bars represent the standard error of the mean.', label='r_pe_cs3')
data_pe_cs3['difficulty']=''
data_pe_cs3['scrambled']=''
data_pe_cs3.ix[(data_pe_cs3['CoI'] == 'emotion-hard') | (data_pe_cs3['CoI'] == 'scrambling-11'), 'difficulty'] = 'hard'
data_pe_cs3.ix[(data_pe_cs3['CoI'] == 'emotion-easy') | (data_pe_cs3['CoI'] == 'scrambling-23'), 'difficulty'] = 'easy'
data_pe_cs3.ix[(data_pe_cs3['scrambling'] == 0), 'scrambled'] = 'no'
data_pe_cs3.ix[(data_pe_cs3['scrambling'] != 0), 'scrambled'] = 'yes'
\end{pycode}
\begin{pycode}[pe_cs4]
pytex.add_dependencies('/home/chymera/src/faceRT/analysis/RTdistribution.py')
pytex.add_dependencies('/home/chymera/src/faceRT/analysis/data_functions.py')
pytex.add_dependencies('/home/chymera/src/faceRT/analysis/gen.cfg')
sys.path.append('/home/chymera/src/faceRT/analysis/')
import RTdistribution
keep_scrambling=(0, 11, 23)
data_pe_cs4 = RTdistribution.main(experiment='11px-4px-5steps_narrow-angle', prepixelation=6, keep_scrambling=keep_scrambling, source='server', make_tight={"pad": 0}, print_title=False)
fig_pe_cs4 = latex_figure(save_fig(fig_width=6.64, fig_height=3), caption='Histogram of reaction times for Hariri-style face matching. Scrambled images for simple visual recognition were preprocessed with a scrambling kernel of \SI{6}{\pixel} and with scrambling cluster. Missed targets have not been counted. For this graphic we have only indexed the '+ ', '.join([str(i) for i in keep_scrambling]) +' pixel scrambling clusters.', label='r_pe_cs4')
\end{pycode}
\begin{pycode}[pe_cs5]
pytex.add_dependencies('/home/chymera/src/faceRT/analysis/signal_detection.py')
pytex.add_dependencies('/home/chymera/src/faceRT/analysis/data_functions.py')
pytex.add_dependencies('/home/chymera/src/faceRT/analysis/gen.cfg')
sys.path.append('/home/chymera/src/faceRT/analysis/')
import signal_detection
data_pe_cs5 = signal_detection.main(experiment='11px-4px-5steps_narrow-angle', prepixelation=6, source='server', elinewidth=1, ecolor='0.3', make_tight={"pad": 0})
fig_pe_cs5 = latex_figure(save_fig(fig_width=6.64, fig_height=3), caption='Error rates for Hariri-style face matching. Scrambled images for simple visual recognition were preprocessed with composite cluster and kernel based scrambling, with a constant kernel of \SI{6}{\pixel} and clusters of the sizes indicated in the graphic (sizes given in pixels). The error bars represent the standard deviation. Bar-plots with zero-height have been slightly indented for better visibility.', label='r_pe_cs5')
data_pe_cs5['difficulty']=''
data_pe_cs5['scrambled']=''
data_pe_cs5.ix[(data_pe_cs5['CoI'] == 'emotion-hard') | (data_pe_cs5['CoI'] == 'scrambling-11'), 'difficulty'] = 'hard'
data_pe_cs5.ix[(data_pe_cs5['CoI'] == 'emotion-easy') | (data_pe_cs5['CoI'] == 'scrambling-23'), 'difficulty'] = 'easy'
data_pe_cs5.ix[(data_pe_cs5['scrambling'] == 0), 'scrambled'] = 'no'
data_pe_cs5.ix[(data_pe_cs5['scrambling'] != 0), 'scrambled'] = 'yes'
\end{pycode}
\begin{pycode}[p_mr]
pytex.add_dependencies('/home/chymera/src/faceOM/analysis/timecourse_dilation.py')
pytex.add_dependencies('/home/chymera/src/faceRT/analysis/data_functions.py')
pytex.add_dependencies('/home/chymera/src/faceRT/analysis/gen.cfg')
sys.path.append('/home/chymera/src/faceOM/analysis/')
import timecourse_dilation
data_p_mr = timecourse_dilation.corr(source='local', make_tight={"pad": 0})
fig_p_mr = latex_figure(save_fig(fig_width=3.5, fig_height=3.5), caption='Correlation between X-axis and Y-axis pupil diameters on measurement data points. Raw data points are shown in faint gray, and per-participant per-timepoint (relative to trial onset) means are shown in green. The regression line is drawn in magenta. To decrease cluttering, only one in three raw data points is plotted.', label='p_mr')
\end{pycode}
\begin{pycode}[ra_rt]
pytex.add_dependencies('/home/chymera/src/faceOM/analysis/rt.py')
pytex.add_dependencies('/home/chymera/src/faceRT/analysis/data_functions.py')
pytex.add_dependencies('/home/chymera/src/faceRT/analysis/gen.cfg')
sys.path.append('/home/chymera/src/faceOM/analysis/')
import rt
data_ra_rt = rt.main(source='local', make_tight={"pad": 0}, make_std=False, make_scrambled_yn=True, elinewidth=1)
fig_ra_rt = latex_figure(save_fig(fig_width=6.64, fig_height=3), caption='Reaction times for Hariri-style face matching. Reaction times for non-response trials were not counted. The error bars represent the standard deviation.', label='ra_rt')
\end{pycode}
\begin{pycode}[ra_er]
pytex.add_dependencies('/home/chymera/src/faceOM/analysis/er.py')
pytex.add_dependencies('/home/chymera/src/faceRT/analysis/data_functions.py')
pytex.add_dependencies('/home/chymera/src/faceRT/analysis/gen.cfg')
sys.path.append('/home/chymera/src/faceOM/analysis/')
import er
data_ra_er = er.main(source='local', make_tight={"pad": 0}, make_std=False, make_scrambled_yn=True, elinewidth=1)
fig_ra_er = latex_figure(save_fig(fig_width=6.64, fig_height=3), caption='Error rates for Hariri-style face matching. The error bars represent the standard deviation. Bar-plots with zero-height have been slightly indented for better visibility.', label='ra_er')
\end{pycode}
\begin{pycode}[r_pd_ta]
import pandas as pd
pytex.add_dependencies('/home/chymera/src/faceOM/analysis/barplot_dilation.py')
pytex.add_dependencies('/home/chymera/src/faceRT/analysis/data_functions.py')
pytex.add_dependencies('/home/chymera/src/faceRT/analysis/gen.cfg')
sys.path.append('/home/chymera/src/faceOM/analysis/')
import barplot_dilation
data_r_pd_ta = barplot_dilation.main(source='local', make_tight={"pad": 0}, make_std=False, make_sem=True, elinewidth=1)
fig_r_pd_ta = latex_figure(save_fig(fig_width=6.64, fig_height=3.2), caption='Bar plots of pupil dilation integrated over the entire time course. The Y-axis values are proportional to the global fixation (inter-trial interval) pupil size. The error bars represent the standard error of the mean.', label='r_pd_ta')
data_r_pd_ta = data_r_pd_ta["Pupil"].groupby(level=(0,2)).mean()
data_r_pd_ta_av = data_r_pd_ta.reset_index()
data_r_pd_ta_av.columns=["CoI","ID","Pupil"]
\end{pycode}
\begin{pycode}[r_pd]
pytex.add_dependencies('/home/chymera/src/faceOM/analysis/timecourse_dilation.py')
pytex.add_dependencies('/home/chymera/src/faceRT/analysis/data_functions.py')
pytex.add_dependencies('/home/chymera/src/faceRT/analysis/gen.cfg')
sys.path.append('/home/chymera/src/faceOM/analysis/')
import timecourse_dilation
data_r_pd = timecourse_dilation.time_course(source='local', make_tight={"pad": 0}, show=["fix", "all", "rt_all"])
fig_r_pd = latex_figure(save_fig(fig_width=6.64, fig_height=3.2), caption='Pupil dilation over time. The values are proportional to the mean pupil dilation size during fixation (the inter-trial interval). Shaded areas represent the standard errors of the mean. The vertical line represents the mean reaction time. Plotted data was down-sampled to \\SI{15}{\\hertz}.', label='r_pd')
data_r_pd_IDmeans = data_r_pd.set_index(["ID"], append=True, drop=True)
data_r_pd_IDmeans = data_r_pd_IDmeans.groupby(level=(0,2)).mean()
\end{pycode}
\begin{pycode}[r_pd_ed1]
pytex.add_dependencies('/home/chymera/src/faceOM/analysis/timecourse_dilation.py')
pytex.add_dependencies('/home/chymera/src/faceRT/analysis/data_functions.py')
pytex.add_dependencies('/home/chymera/src/faceRT/analysis/gen.cfg')
sys.path.append('/home/chymera/src/faceOM/analysis/')
import timecourse_dilation
data_r_pd_ed1 = timecourse_dilation.time_course(source='local', make_tight={"pad": 0}, show=["easy", "hard", "rt_e", "rt_h"])
fig_r_pd_ed1 = latex_figure(save_fig(fig_width=6.64, fig_height=3.2), caption='Pupil dilation over time for easy and difficult matching. The values are proportional to the mean pupil dilation size during fixation (here 1.0). Shaded areas represent the standard errors of the mean. Vertical lines represent mean reaction times. Plotted data was down-sampled to \\SI{15}{\\hertz}.', label='r_pd_ed1')
data_r_pd_ed1_lm = data_r_pd_ed1.reset_index()
data_r_pd_ed1_lm = data_r_pd_ed1_lm[(data_r_pd_ed1_lm["CoI"] =="easy")|(data_r_pd_ed1_lm["CoI"] =="hard")&(data_r_pd_ed1_lm["measurement"] >= 120)]
\end{pycode}
\begin{pycode}[r_pd_ed2]
pytex.add_dependencies('/home/chymera/src/faceOM/analysis/timecourse_dilation.py')
pytex.add_dependencies('/home/chymera/src/faceRT/analysis/data_functions.py')
pytex.add_dependencies('/home/chymera/src/faceRT/analysis/gen.cfg')
sys.path.append('/home/chymera/src/faceOM/analysis/')
import timecourse_dilation
data_r_pd_ed2 = timecourse_dilation.discrete_time(make_tight={"pad": 0}, show=["easy", "hard"], sample_size = 30)
fig_r_pd_ed2 = latex_figure(save_fig(fig_width=6.64, fig_height=3.2), caption='Pupil dilation over time for easy and difficult matching. The values are proportional to the mean pupil dilation size during fixation (here 1.0). Shaded areas represent the standard errors of the mean. Vertical lines represent mean reaction times. Plotted data was down-sampled to \\SI{15}{\\hertz}.', label='r_pd_ed2')
data_r_pd_ed2 = data_r_pd_ed2.reset_index()
data_r_pd_ed2 = data_r_pd_ed2[(data_r_pd_ed2["CoI"] == "easy") | (data_r_pd_ed2["CoI"] == "hard")]
\end{pycode}
\begin{pycode}[r_pd_hf1]
pytex.add_dependencies('/home/chymera/src/faceOM/analysis/timecourse_dilation.py')
pytex.add_dependencies('/home/chymera/src/faceRT/analysis/data_functions.py')
pytex.add_dependencies('/home/chymera/src/faceRT/analysis/gen.cfg')
sys.path.append('/home/chymera/src/faceOM/analysis/')
import timecourse_dilation
data_r_pd_hf1 = timecourse_dilation.time_course(source='local', make_tight={"pad": 0}, show=["happy", "fearful", "rt_ha", "rt_fe"])
fig_r_pd_hf1 = latex_figure(save_fig(fig_width=6.64, fig_height=3.2), caption='Pupil dilation over time for emotional faces matching. The values are proportional to the mean pupil dilation size during fixation (here 1.0). Shaded areas represent the standard errors of the mean. Vertical lines represent mean reaction times. Plotted data was down-sampled to \\SI{15}{\\hertz}.', label='r_pd_hf1')
data_r_pd_hf1_IDmeans = data_r_pd_hf1.set_index(["ID"], append=True, drop=True)
data_r_pd_hf1_IDmeans = data_r_pd_hf1_IDmeans.groupby(level=(0,1)).mean()
\end{pycode}
\begin{pycode}[r_pd_hf2]
pytex.add_dependencies('/home/chymera/src/faceOM/analysis/timecourse_dilation.py')
pytex.add_dependencies('/home/chymera/src/faceRT/analysis/data_functions.py')
pytex.add_dependencies('/home/chymera/src/faceRT/analysis/gen.cfg')
sys.path.append('/home/chymera/src/faceOM/analysis/')
import timecourse_dilation
data_r_pd_hf2 = timecourse_dilation.discrete_time(make_tight={"pad": 0}, show=["happy", "fearful"], sample_size = 30)
fig_r_pd_hf2 = latex_figure(save_fig(fig_width=6.64, fig_height=3.2), caption='Pupil dilation over time for easy and difficult matching. The values are proportional to the mean pupil dilation size during fixation (here 1.0). Shaded areas represent the standard errors of the mean. Vertical lines represent mean reaction times. Plotted data was down-sampled to \\SI{15}{\\hertz}.', label='r_pd_hf2')
data_r_pd_hf2 = data_r_pd_hf2.reset_index()
data_r_pd_hf2 = data_r_pd_hf2[(data_r_pd_hf2["CoI"] == "happy") | (data_r_pd_hf2["CoI"] == "fearful")]
\end{pycode}
\begin{pycode}[r_pd_es1]
pytex.add_dependencies('/home/chymera/src/faceOM/analysis/timecourse_dilation.py')
pytex.add_dependencies('/home/chymera/src/faceRT/analysis/data_functions.py')
pytex.add_dependencies('/home/chymera/src/faceRT/analysis/gen.cfg')
sys.path.append('/home/chymera/src/faceOM/analysis/')
import timecourse_dilation
data_r_pd_es1 = timecourse_dilation.time_course(source='local', make_tight={"pad": 0}, show=["emotion", "scrambled", "rt_em", "rt_sc"])
fig_r_pd_es1 = latex_figure(save_fig(fig_width=6.64, fig_height=3.2), caption='Pupil dilation over time for matching of either emotional or scrambled faces. The values are proportional to the mean pupil dilation size during fixation (here 1.0). Shaded areas represent the standard errors of the mean. Vertical lines represent mean reaction times. Plotted data was down-sampled to \\SI{15}{\\hertz}.', label='r_pd_es1')
data_r_pd_es1 = data_r_pd_es1.reset_index()
data_r_pd_es1[(data_r_pd_es1["CoI"] =="fearful")]["CoI"] = (data_r_pd_es1[(data_r_pd_es1["CoI"] =="fearful")]["CoI"]+data_r_pd_es1[(data_r_pd_es1["CoI"] =="happy")]["CoI"])/2
data_r_pd_es1.ix[(data_r_pd_es1["CoI"]=="fearful"), "CoI"] = "emotion"
data_r_pd_es1_lm = data_r_pd_es1[(data_r_pd_es1["CoI"] =="scrambled")|(data_r_pd_es1["CoI"] =="emotion")&(data_r_pd_es1["measurement"] >= 120)]
\end{pycode}
\begin{pycode}[r_pd_es2]
pytex.add_dependencies('/home/chymera/src/faceOM/analysis/timecourse_dilation.py')
pytex.add_dependencies('/home/chymera/src/faceRT/analysis/data_functions.py')
pytex.add_dependencies('/home/chymera/src/faceRT/analysis/gen.cfg')
sys.path.append('/home/chymera/src/faceOM/analysis/')
import timecourse_dilation
data_r_pd_es2 = timecourse_dilation.discrete_time(make_tight={"pad": 0}, show=["happy", "fearful"], sample_size = 30)
data_r_pd_es2 = data_r_pd_es2.reset_index()
data_r_pd_es2[(data_r_pd_es2["CoI"] =="fearful")]["CoI"] = (data_r_pd_es2[(data_r_pd_es2["CoI"] =="fearful")]["CoI"]+data_r_pd_es2[(data_r_pd_es2["CoI"] =="happy")]["CoI"])/2
data_r_pd_es2.ix[(data_r_pd_es2["CoI"]=="fearful"), "CoI"] = "emotion"
data_r_pd_es2 = data_r_pd_es2[(data_r_pd_es2["CoI"] == "emotion")|(data_r_pd_es2["CoI"] == "scrambled")]
\end{pycode}
%END FIGURES AND DATA
