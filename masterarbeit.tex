\documentclass[a4paper,romanian,english]{book}
\usepackage[utf8]{inputenc}
\usepackage[T1]{fontenc}
\usepackage{babel}
\usepackage{graphicx}
\usepackage{multicol}
\usepackage{color}
\usepackage{float}
\usepackage{wasysym}
\usepackage{verbatim} 
\usepackage[labelfont={bf}, margin=0.8cm]{caption}
\usepackage{fancyhdr}
\usepackage[iso, english]{isodate}
\usepackage{dashrule}
\usepackage{siunitx} %pretty measurement unit rendering
\usepackage{catchfilebetweentags} % used to import the REAMDE w/o the reST header
\usepackage{courier} % For prettier monospace fonts (used for code)
\usepackage{color,listings} % background color highlighting for code
\usepackage[parfill]{parskip} % blank lines instead of indents to separate paragraphs
\usepackage[hidelinks]{hyperref} % Makes URLs, DOIs, and page links work. w/o colored bracketing.
\usepackage[margin=2.4cm, top=2.5cm]{geometry}% Makes margins narrower.


\renewcommand{\theequation}{\arabic{equation}}
\widowpenalty1000 % no widows
\clubpenalty1000 % no ohans

\definecolor{vlg}{gray}{0.90}
\definecolor{lg}{gray}{0.60}
\definecolor{mg}{gray}{0.40}


\makeatletter
\renewcommand\chapter{
	\thispagestyle{headings}
	\if@openright\cleardoublepage\else\clearpage\fi\global\@topnum\z@
	\secdef\@chapter\@schapter
	}
\makeatother


\makeatletter
\newenvironment{tablehere}
{\def\@captype{table}}
{}
\newenvironment{figurehere}
{\def\@captype{figure}}
{}
\makeatother

\makeatletter
\def\maketitle{%
    \setlength{\parindent}{0cm}
    \null
    \pagestyle{empty}%
    \vfill
    \begin{center}\leavevmode
    \normalfont{}
    \vskip 1.5cm
    {\Large \textbf{\@title}\par}%
    \vskip 3cm
    {\Large Master Thesis\par}
    \vskip 0.5cm
    {\Large Presented to the Faculty of Biosciences\\at the Ruprecht-Karls-Universität Heidelberg\par}
    \vskip 5cm
    {\Large \@author\par}%
    {\Large \the\year\par}%
    \end{center}%
    \vfill
    \null
    \cleardoublepage{}
    \null
    \vfill{}
    \large
    This thesis was written at the \textbf{Central Institue of Mental Health} at the University of Heidelberg in the period of \textbf{2013-06-06} to \textbf{\today} under supervision of \textbf{Prof. Dr. Peter Kirsch}.
    \vskip 3cm
    1$^{st}$ \hspace{0.085cm}Appraiser: Prof. Dr. Rainer Spanagel \hfill Institute: Central Institute of Mental Health\\    
    2$^{nd}$ Appraiser: Prof. Dr. Daniel Durstewitz \hfill Institute: Central Institute of Mental Health\\
    \vskip 3cm
    I herewith declare that I wrote this Master thesis independently under supervision and used no other sources and aids than those indicated.
    \vskip 3.5cm
    \hdashrule{4cm}{1pt}{1pt 1pt}\hspace{5cm}\hdashrule{5cm}{1pt}{1pt 1pt}
    \\
    Date\hspace{8.2cm}Signature
    \vskip 2cm
    \hdashrule{\textwidth}{2pt}{1pt 1pt} 
    \vskip 2.5cm
    The thesis has to contain a summary in English.
    \vfill
    \null{}
    \setlength{\parindent}{0.55cm}
    \cleardoublepage
}

\makeatother
\title{“Neuronal Correlates of Occulometric Parameters in Face Recognition”}
\author{Horea-Ioan Ioanãş}

\begin{document}
\maketitle

\pagestyle{headings}
\tableofcontents{}
\setcounter{page}{1}
\setlength{\parindent}{0pt}
\setlength{\hangindent}{0pt}


\chapter{Summary}
    \section{English}
	\CatchFileBetweenDelims{\readme}{README.rst}{.. engl}{.. engl>} %adds file content between the specified tags 
	\readme
    \clearpage
    \section{German}
	\CatchFileBetweenDelims{\readme}{README.rst}{.. ger}{.. ger>} %adds file content between the specified tags 
	\readme

\chapter{Background}
    \section{Pupillometry}\label{sec:b_p}
	Pupillometry refers to the long-standing physiological method of measuring pupil diameter.
	Sporadic studies involving pupil diameter measurements have been performed as early as the beginning of the XIX. century \cite{Loewenfeld1958},
	with the establishment of a scientific community around pupillometry following about oone and a half centuries later.
	The early emergence of pupillometric research owes to the innovation of infra-red imaging to study the eye \cite{Dubois1955}, around the middle of the XX. century (originally for ophthalmologic purposes, where pupillometry also continues to be used \cite{Thompson2012}).
	This development made it possible to measure pupil kinetics without interference from the measurement equipment itself - and remains a standard for even the newest devices of our day \cite{Bradley2010}.
	
	Pupillometry has been in documented use in psychology since 1960, when Hess and Polt published what was arguably the seminal paper of psychophysiological pupillometry in the journal \textit{Science} \cite{HESS1960}.
	In this publication they reported presenting images of partly nude adults and babies to a small sample of participants, and observing increased pupil dilation in males to stimuli depicting women, and increased pupil dilation in women to stimuli depicting either males or babies.
	In light of this manifold interpretable finding, research over the next decade has focused on making the case that pupil dilation is a marker of attraction, or well-being.
	
	Many early publications present varying concepts related to the initial publication by Hess and Polt, and report a robust modulation of pupil diameter by sexual attraction \cite{Goldwater1972, HESS1965} or positive affect \cite{Nunally1967, Bradshaw1967}.
	Even some of these early papers, however, hinted at a broader phenomenon underlying neuropsychological control of pupil size.
	This broader phenomenon was initially termed “general activation”\cite{Nunally1967}, and has over the years been more accurately identified as “cognitive load”\cite{Zekveld2011}, “attention” \cite{Wykowska2013,Kraemer2000}, or - most recently - “control state” \cite{Hayes2013}.
	Of late, researchers have tentatively tried to establish pupillometry as an indicator of neuronal activity in the Locus Coeruleus (LC; more on this in section~\ref{sec:b_lc}) \cite{Gilzenrat2010}.
	In the research herein presented, we have sought to test the viability of pupillometry as an assessment not of a neuropsychological phenomenon, but of a neurological phenomenon - 
	complementing mostly behavioural previous approaches\cite{Gilzenrat2010,Granholm2004} with concomitant brain imaging.
	
    \section{Pupil Dilation Response}
	Pupilary dilation (mydriasis) is most prominently known as a physiological reaction to low light conditions \cite{Ellis1981}. 
	However, a broad range of mental states - 
	such as alertness\cite{Yoss1970}, arousal\cite{Bradshaw1967}, fatigue\cite{Morad2000}, and surprize\cite{Preuschoff2011} -
	have also been shown to modulate pupil dilation.
	Additionally, application of numerous substances (topically and/or systemically) is known to affect pupil dilation \cite{Theofilopoulos1988, Bye1979, Phillips2000}; to the extent that the method is used to assess drug effects in laboratory animals \cite{Murray1981}.
	Disregulation in tonic pupil dilation has also been implicated in non-neurodegenerative diseases - most notably in autism spectrum disorders \cite{Anderson2013}.
	
	The direct control mechanism of pupil dilation (or contraction for that matter) are two sets of smooth muscles located in the iris: the iris dilator muscle and the iris sphincter muscle.
	\begin{itemize}
	    \item The iris dilator is a radial muscle located anterior of the pigment epithelium of the iris, and chiefly responsible for iris dilation.
	    This muscle is innervated by the sympathetic nervous system and is controlled via noradrenergic stimulation to its $\alpha_1$-adrenergic receptors \cite{vanAlphen1976}.
	    \item The iris sphincter muscle is located anterior of the iris dilator muscle and chiefly responsible for iris contraction.
	    Its innervation stems from the parasympathetic system and is mainly mediated by M$\mathsf{_3}$-muscarinic acetylcholine receptors \cite{Woldemussie1993,Taylor1974}.
	\end{itemize}
	It should be noted that counterbalanced neurotransmitter effects (such as direct cholinergic inhibition of the iris dilator muscle) have also been documented \cite{Yoshitomi1985}.
	
	In light of these characteristics it is not surprising that topical application of drugs can affect pupil dilation.
	This phenomenon is especially robust and well-documented for cholinergic agonists \cite{Smith1978} and antagonists \cite{Gambill1967} -
	to the extent that pupil dilation is used as an assay for in vivo anticholinergic effects \cite{Bye1979}.
	Serotonin (5-HT) receptors have also been shown to affect pupil dilation - both by topically applied serotonin \cite{KOELLA1962}, or by systemically applied receptor agonists \cite{Yu2004}.
	Systemic application of 5-HT agonists (in rats) has furthermore been shown to elicit mydriasis not via peripheral 5-HT receptors, but via CNS 5HT$\mathsf{_{1A}}$ receptors.
	
	Considerable attention was received by the hypothesis that CNS 5-HT (in liaison with depression, and a long-standing interest of pupillometrists in affect - detailed under section~\ref{sec:b_p}) may be chiefly implicated in controlling pupil dilation.
	As a most prominent example of pupil-dilation and serotonin interaction, antidepressants of the selective serotonin re-uptake inhibitor (SSRI) class are widely documented to cause tonic mydriasis \cite{Fitzgerald2013,Klein-Schwartz2012}.
	
	Relying on this established phenomenon, at least one study has tried to investigate the value of pupil dilation as a convenient, semi-quantitative 5-HT biomarker \cite{Noehr-Jensen2009}.
	The study found an escitalopram-correlated reduction of light-evoked pupil constriction, but no overall effect of escitalopram (which the authors use as a proxy for systemic 5-HT) on pupil dilation.
	The authors conclude that pupillometry can, thus, not be used as a 5-HT biomarker.
	In this context we would, however, like to observe that theirs was a chronic test for tonic dilation.
	In fact, the modulation of the pupillary light reflex found by the authors may even indicate that the outcome of their study would have been quite different, had it employed pupillometry to assay phasic dilation related to acute events (such as for instance in a behavioural task).
	In any case, the results of this study conflict with the often reported dose-dependent pupil dilator activity of SSRIs \cite{Nielsen2010,Fitzgerald2013,Klein-Schwartz2012};
	and need not dismiss all hopes of pupillometry as a serotonin biomarker.
	
	Evidence has mounted considerably faster and more consistently on establishing pupil diameter as a biomarker for norepinephrine (NE).
	In light of this, it may be hypothesized that 5-HT works upstream of NE in controlling pupil dilation; and this may lead to the opaquer state of research concerning the correlation of 5-HT and pupil dilation compared to the correlation of NE and pupil dilation.
	Research into the pupil dilation effects of 3,4-methylenedioxymethamphetamine (MDMA) administered in the presence of varying monoamine reuptake inhibitors and monoamine agonists has - however - shown that both 5-HT and NE act on pupil dilation independently of each other \cite{Hysek2012}.
	Additionally, 5-HT increase - as mediated by SSRI administration - has been documented in both anxiety \cite{Blier2007} and cardiovascular \cite{Barton2007} studies to reduce sympathetic (noradrenergic) activation over time. 
	Though these findings cannot completely exclude synergistic effects, we believe it reasonable to investigate NE and 5-HT mediated pupil dilation separately.
	
	A noradrenergic sensitivity of pupil dilation has been demonstrated in studies with the $\alpha_2$-adrenergic receptor agonist clonidine, and the (more broad spectrum) $\alpha_2$-adrenergic receptor antagonist yohimbine \cite{Phillips2000}. 
	As would be expected from the aforementioned findings (and with the $\alpha_2$-adrenoceptor being mainly an autoreceptor), clonidine lead to a decrease in pupil diameter while yohimbine lead to an increase; co-administration showed that these effects counteract each other.
	
	Strengthening the association between the NE system and pupil dilation, evidence from electrophysiology studies performed in macaques \cite{Rajkowski1994} shows a strong correlation and time-lock between neuronal firing in the main noradrenergic nucleus of the brain (the locus coeruleus) and pupil dilation.
	Researchers have sought to further validate the connection between pupil dilation and LC neuronal activity by testing the correlation between pupil response and phemonena which are considered markers of LC activity.
	
	One accessible avenue of doing so is pupil-dilation to LC matching via behavioural patterns \cite{Gilzenrat2010}.
	In the referenced study authors observe that tonic pupil dilation correlates with exploratory behaviour and task disengagement, while phasic pupil dilation correlates with increased task engagement; they argue that this closely mimics documented \cite{Aston-Jones2005} changes in behaviour upon tonic or phasic LC activity.
	In the review of this study it is very important to note that pupil dilation correlates \textit{not} with task difficulty per se, but with task engagement - meaning that if a task is to difficult to fulfil, pupil diameter may not see a phasic increase.
	Interestingly, similar pupil dilation kinetics have been found during off-line thought \cite{Smallwood2011}, and have been again put into liaison with noradrenergic LC activity (as this is known to play a pivotal role in attention control - more under section~\ref{sec:b_lc}).   
	
	Other researchers have tried to test the correlation between physiological phenomena which are strongly suspected of being LC-mediated, and make the case that good correlation between these strengthens the probability of a common cause (the only candidate for which would be LC activation) \cite{Murphy2011}.
	In the annotated example the authors test pupil dilation kinetics against the P3 (also P300) event-related potential (known for many decades for its role in decision-making \cite{CHAPMAN1964}) and observe a strong and positive correlation.
	
	Tests using more direct assessments of LC activation in humans are scarce, traditionally due to methodological restrictions.
	The advent of fMRI has also had less impact here than in research on other brain areas due to the nonstandard blood-oxygen level dependent (BOLD) response patterns of brainstem structures and the small size of the nucleus \cite{Astafiev2010}.
	
    \section{Locus Coeruleus}\label{sec:b_lc}
	The Locus Coeruleus (LC) is a bilateral brainstem nucleus located in the pons, close to the superior cerebellar pedunculi.
	Its name and original identification base on the (very faintly) blue appearance of the nucleus in fresh brain tissue \cite{Maeda2000}.
	The distinctive blue colour is attributed to monoamine (in this case noradrenaline) polymerization and is analogous to the colouring of other brainstem nuclei such as the substantia nigra \cite{Mai2011}.
	The right nucleus of the LC has been documented to comprise approx. 33.600 catecholaminergic cells, while the cell count of the left nucleus lies at about 39.100 \cite{Mouton1994}.
	While post mortem dissection-based coordinates for the LC abound, we believe that in the light of prospective fMRI study, in-vivo magnetic resonance determined coordinates are the best guideline \cite{Keren2009}.
	
	LC activity has been originally identified in response to various painful stimuli, has gradually been expanded to include first non-painful threat signals \cite{Grant1984}, and subsequently many more events.
	The LC is known chiefly for its role in noradrenergic neuronal transmission \cite{Benarroch2009}, and based on this has been implicated in a wide range of phenomena, exceeding but including:
	\begin{multicols}{2}
	    \begin{itemize}
		\item Anxiety \cite{Weiss1994}
		\item Attention \cite{Benarroch2009}
		\item Autism \cite{Mehler2009}
		\item (Emotional) Arousal \cite{Bangasser2011,Benarroch2009}
		\item Depression \cite{Bangasser2011, Weiss1994}
		\item Drug Use \cite{Samuels2008}
		\item Neural plasticity \cite{Benarroch2009}
		\item Neurodegenerative Symptoms \cite{Samuels2008,Gesi2000}
		\item Psychological Stress \cite{Bangasser2011,Benarroch2009}
		\item Posture and Balance \cite{Benarroch2009}
	    \end{itemize}
	\end{multicols}
	
	From the point of view of pupillometry-based neuronal activity assay, the LC is of interest due to the documented correlation and time-lock between pupil dilation and LC neuron activity in macaques \cite{Rajkowski1994}.
	Further electrophysiological findings in macaques coming from the same authors \cite{Rajkowski2004} found a strong correlaton and time-lock between reaction times in behavioural trials and phasic firing of single neurons in the LC.
	Homogeneous results all over the LC additionally prompted the authors to conclude that - at least in view of this paradigm - LC neurons are physiologically similar across the nucleus \cite{Rajkowski2004}.
	
	The noradrenergic activity of the LC is chiefly implicated in attention control \cite{Aston-Jones1994,Gabay2011}, with pupil dilation also mimicking the inferred LC on-line activity in man \cite{Gabay2011}.
	A well-documented model for LC noradrenergic control of attention has emerged under the name of “adaptive gain”.
	This model proposes that noradrenergic neurons of the LC serve to increase activity contrast between certain cortical units to afford more downstream effect to some and less to others; this contrast increase is formally termed “gain”  \cite{Aston-Jones2005}.
	Evidence for this model has emerged from behavioural task engagement and disengagement, evaluated in light of LC neuronal efferences to the orbitofrontal and anterior cingulate cortices (OFC and AFC respectively) \cite{Aston-Jones2005}; both of these structures being well-known for their role in behavioural control \cite{Baxter2013,Kerns2004}.
	The adaptive gain model has also found supporting evidence in humans, where global fluctuations in functional connectivity have been shown to correlate with task difficulty and task-effected pupil dilation \cite{Eldar2013}.
	
	Pharmacological studies in humans have shown that the effects of the wakefulness-promoting drug \cite{Engber1998} modafinil can be reversed by clonidine, presumably due to an LC mediated mechanism of action for modafinil \cite{Hou2005}.
	A high-profile fMRI study published in the journal \textit{Science} reiterated this hypothesis and identified an LC activity shift upon modafinil administration via functional imaging.
	The accuracy of the imaging has, however, subsequently been drawn into question, with commentators \cite{Astafiev2010} pointing at discrepancies between the LC and the brain stem activation coordinates reported.  
	BOLD LC signals have been described as difficult to interpret when using response magnitudes as the main feature, and authors recommended a more time-course sensitive signal evaluation method \cite{Astafiev2010}.
	Though no such recommendation was made, we believe applying pupil diameter as a regressor would well complement previous studies. 

	Summarizing the current state of research, we believe that due to the wide-ranging processes which the LC is causally involved in controlling, its study is relevant for a broad number of neuropsychology subfields.
	Especially in liaison with the purported possibility of approximating LC activation via pupillometry \cite{Gilzenrat2010,Murphy2011}; further research could be conducive to better techniques for monitoring and diagnosing affect, attention, and anxiety disorders, as well as gauging or predicting the effectiveness of drugs or therapy.
	Due to the portability \cite{Bradley2010} and (comparatively) reduced costs of pupillometric equipment, we believe the well-informed use of pupillometry in neuropsychology would be of great benefit to personalized treatment and diagnosis.
	We also envision a utility herefor in the study of self-administered (recreational or medical) drugs in natural settings - a research field where equipment portability is of utmost importance. 

\iffalse
    \section{Open Science}\label{sec:b_os}
	The format of scientific publishing has become a highly debated topic in recent years - as exemplified by \textit{Nature}'s landmark “Peer Review Trial” of 2006 \cite{Nature-debate2006}.
	Much of the discussion in the scientific community concerns open access \cite{VanNoorden2013,Parker2013}, and the more controversial open peer review.
	While these discussions are formally dealing with publishing formats other than the classical “thesis”, we hold the concerns of the authors relevant to all forms of knowledge presentation.
	
	In creating this document we have kept the following considerations in mind, and used all of the below mentioned software.
	\subsection{Reproducible Data}\label{sec:b_rr}
	    Data reproducibility 
	\subsection{Reproducible Analysis}\label{sec:b_ra}
	    In addition to reproducing the raw material of research - the data - transparent research also relies on reproducibility of analysis \cite{Peng2009}.
	    Especially in fields where presented results rely on numerous statistical manipulations (as many sub-fields of modern neuroscience do) this concern is of utmost relevance.
	    	    
	\subsection{Reproducible Documents}\label{sec:b_rd}
	    The final form of a scientific project - the “publication” or the “document” represents the condensed form of knowledge and as such integrates the most work and is the most difficult to reproduce \cite{Schwab2000}.
	    
	    Reproducibility of a document does not refer merely to recompiling static text (for instance from \colorbox{vlg}{\texttt{.tex}} to \colorbox{vlg}{\texttt{.pdf}}), but rather recompiling a dynamic representation of the results.
	    Therefore, a reproducible document is only to be understood as an additional layer on top of reproducible data analysis, and should avoid wherever possible inclusion of static (copy-paste) measures obtained from the data. 
	    
	    A viable solution herefor is making the document \textit{live} and creating figures and numeric representations of the data at compile time.
	    A prerequisite 
	    
	    
	    For piping both \textit{live} python data frames and \textit{live} matplotlib figures to our document, we have found the PythonTeX \cite{Poore2013} package more than adequate.
	    Live data within a reproducible document can be used for additional micro-analysis (such as t-tests, ANOVA, or linear models).
	    Wrappers for statistical functions - as for instance the interstats package \cite{interstats} - are a convenient way of doing so.
	    More complex statistical methods (such as linear models) may require passing tables from the backend (which can be R\cite{R}, for example) to the \LaTeX\ compiler.
	    There are also a wealth of packages providing direct (and wrappable) R-to-\LaTeX\ piping.
	    Two of the most prominent such packages are texref \cite{Leifeld2013} and stargazer \cite{Hlavac2013}.
\fi

\chapter{Methods}\label{sec:m}
    \section{Participants}\label{sec:m_p}
	Participants were recruited via flyers and advertisements pinned to information boards in multiple locations at the University of Mannheim and the University of Heidelberg.
	As part of this study, a total number of 12 participants have been tested; 4 male and 8 female.
	Participants were aged 18-29, and had no a priori documented psychiatric or neurologic conditions.
	
	They were screened for fMRI testing and presented with consent forms in accordance with the ethical guidelines at the Central Institute of Mental Health. 
    \section{Behavioural Test}\label{sec:m_bt}
	In our current study we are exploring neuronal function in visual emotional perception.
	Our behavioural test of choice to tap into these systems relies on visual recognition and subsequent matching of emotion.
	We call our paradigm a Hariri-style behavioural test, in acknowledgement of the seminal implementations of a similar paradigm by Hariri and colleagues \citep{Hariri2000,Hariri2003}.
	    
	The paradigm consists of a triangular arrangement of three faces - with the bottom two aligned horizontally and the upper face displaced vertically and equidistant from both bottom faces.
	The behavioural test prompts the participant to observe a certain feature (in our case emotion, or - as described in section~\ref{sec:m_vs_si} - pixel arrangement) of the top face, and to select of the bottom faces the one repeating that feature.
	
	In our case bottom face selection was done via the arrow keys of a keyboard (in the preliminary experiments reported under section~\ref{sec:pe_r}) and via a remote control (in the fMRI trials reported under section~\ref{sec:m_fmri}).
	The participants received no sort of feedback after reacting to the individual items.
	
	The images were continuously displayed independently of input for \SI{5}{\second} or \SI{4.5}{\second} (in the preliminary experiments), and for \SI{4}{\second} (in the fMRI trials).
	Intertrial intervals showed a point (in the preliminary experiments) or a cross (in the fMRI experiments) and lasted for \SI{2}{\second} or approximately \SI{4}{\second} respectively.
	The higher inter-trial interval duration for the fMRI experiments was chosen to accomodate for two volume acquisitions.
	
	For the fMRI trials we have employed fixed trial sequences to increase signal detection power.
	These trial sequences were calculated using the Optseq2 \citep{optseq} software package, from a list with permitted and counterbalanced stimulus compositions computed by our Pystim \citep{pystim} script.
	Trial durations (as detailed under section~\ref{sec:pe_m_bt_cs}) have been fixed at \SI{4}{\second}, and inter-trial intervals have been jittered around that value with a maximum amplitude of \SI{2}{\second}.
	These trials were performed after an anatomical measurement and a different behavioural test - collectively lasting approximately \SI{20}{\minute}.
	
	During the above we take measures for valence (correct or incorrect) of responses, and for reaction times.
	
	\begin{wrapfigure}{r}{0.45\textwidth}
	  \centering
	    \scalebox{0.55}{\includegraphics{./img/vis_field.jpg}}
	    \caption{Perimetric map of the human field of view \citep{Ruch1960}.
	    For measurement the head and eyes were fixed, with the fovea pointing at \SI{0}{\degree} on the cross-hairs.
	    The white area affords binocular vision, the black area is completely outside the field of view.}
	    \label{fig:m_b_1}
	    \vspace{-1.0cm}
	\end{wrapfigure}
	
	In contrast to other formulations\citep{Hariri2000,Hariri2003} of the same paradigm (which keep all 3 stimuli equidistant),
	we have decided to adapt the positions of stimulus images to what we believe is a better fit for the human field of view.
	Simply aligning the midpoints of stimulus images to the vertices of an equilateral triangle presents some issues:
	\begin{itemize}
	    \item Not accounting for the aspect ratio of the stimulus images lessens control over the shape and size of the resulting populated visual window (pVW).
	    \item Horizontal and diagonal distances are not registered identically by human vision \citep{TerryBahill1975}.
	\end{itemize}
	
	We tackled the above by adapting the pVW to the normal human field of vision aspect ratio.
	To do so, we took the outermost non-neutral pixels (rather than the centers of the stimulus images) and used these as pVW delimiters in deciding where to position our images.
	
	For the following distance specifications we have defined the unit “u” as the height of our stimulus images.
	Seeing as our stimulus images have a fixed aspect ratio their width is thus consistently \SI[parse-numbers = false]{\frac{3}{4}}{u}.
	
	The static human field of view has a complex shape modulated amongst others by facial features (as seen in figure~\ref{fig:m_b_1}).
	If the outermost light paths fitting figure~\ref{fig:m_b_1} were to be circumscribed by a planar mask in the shape of the rectangle, its height would span approximately \SI{130}{\degree} and its width \SI{180}{\degree}.
	This gives the rectangular fit for the field of view a width-to-height ratio of 0.72(2).
	Given the data - and for numerical convenience - we have chosen to position the midpoints of our stimulus images as follows:
	\begin{itemize}
	    \item The distance between the two lower image midpoints is \SI{2}{u}.
	    \item The vertical offset of the top image midpoint from the horizontal line connecting the bottom image midpoints is \SI{1}{u}
	\end{itemize}
	This gives our pVW a total height of \SI{2}{u} and a total width of \SI[parse-numbers = false]{\frac{11}{4}}{u}.
	The resulting width-to-height ratio of 0.(72) was deemed a sufficiently close match for the afore mentioned human field of view (as approximated by a rectangle).
	We thus have reason to believe that our stimulus item distribution is superior to the one implemented by Hariri and colleagues, and we encourage further users of Hariri-style paradigms to note our observations.  
	
	In addition to field of view shape, saccades are an other parameter which offers valuable information on the accuracy and speed with which the eyes track distance (e.g. across a monitor).
	We are aware that our simplified approximation still does not quantitatively account for vertical saccades being slower \citep{TerryBahill1975} and less accurate \citep{Collewijn1988} than horizontal saccades, 
	nor for the fact that upward saccades tend to undershoot while downward saccades tend to overshoot \citep{Collewijn1988};
	however, integrating accurate metrics for these kinetics and determining their relevance for Hariri-style paradigms differs in focus from and far exceeds the scope of our current study.
	
	Our resulting pVW was scaled to fit the available display field of the monitor and tilted mirror set-up in our MRI scanner (more on this in section~\ref{sec:m_om_et}).
	The dimensions (specified in degrees of visual angle) are proportional to the ones mentioned above, scaled for u = \SI{4.24}{\degree}.
	The midpoint of the pVW (at \SI{0.5}{u} downward from the midpoint of the top image) was aligned to the midpoint of the monitor.
	A picture of this visual set-up can be seen in figure~\ref{fig:m_b_2}.
	
	\begin{figure}[!h]
	    \centering{\includegraphics[width=0.9\textwidth]{./img/01F_FE_C100_em100_composite.jpg}}
	    \caption{Sample image from the emotional face matching trials in our Hariri-style behavioural test.}
	    \label{fig:m_b_2}
	\end{figure}
		    
    \section{Stimulus Images}
	\subsection{Emotional Faces}\label{sec:m_vs_ef}
	    The emotional face stimuli required by our behavioural test paradigm (see section~\ref{sec:m_bt}) have been sourced from the NimStim set of facial expressions \citep{Tottenham2009}.
	    We have decided to use different emotion intensities in order to better describe step-wise effects of emotion perception and/or emotion recognition for either pupil dilation time courses, brain activation, reaction times - or interactions thereof.
	    
	    Our emotion intensities of choice are \SI{40}{\percent} and \SI{100}{\percent} - these being referred to as weak (“hard” in the sense of recognition) and strong (“easy” in the sense of recognition) emotion images respectively.
	    The semi-quantitative measure of emotion (in percent) stems from the image morphing used to obtain these images - \SI{40}{\percent} happy images are thus a 4:6 linear combination of \SI{0}{\percent} happy and \SI{100}{\percent} happy eigenfaces \citep{Zhang2008} respectively.
	    
	    The choice of \SI{40}{\percent} was made to obtain the highest possible activation variance both in brain areas presenting discrete (amygdala) or continuous (superior temporal sulcus) sensitivity to increasing emotional concentrations \citep{Harris2012}.
	\subsection{Scrambled Images}\label{sec:m_vs_si}
	    In order to define a pattern recognition (baseline) condition we have decided to use trials in which visual input consists of the same pixels as in the emotional trials.
	    This is especially relevant to pupillometry, for which we would like to ensure condition-independent luminosity.
	    
	    To disentangle pattern identification from semantic (emotion) processing we decided to use “scrambled” images - in which pixels from the emotional faces are permuted so as to obfuscate emotional expressions and facial features.
	    The recognition trials would thus entail matching identical permutations of the same picture.
	    For these trials the template and target are the exact same image.
	    The distractor is a different permutation of the pixels which constitute the source image (the image which template/target were also computed from). 
	    
	    Corresponding to the emotion recognition trials, we include two sets of scrambled image trials: “easy” and “difficult”.
	    These are to act as baselines for easy and difficult emotion recognition trials respectively.
	    Recognition difficulty of the scrambled image trials should scale proportionally with that of the easy and difficult emotion recognition trials (detailed under section~\ref{sec:m_vs_ef}).
	    To achieve this, scrambling has to be progressively complex.
	    We assume that scrambling of images via progressively smaller sub-sections (further referred to as “clusters”) makes the images progressively more difficult to identify and match to other copies.
	    This rationale determined the nature of our scrambling algorithms (detailed below) and was also experimentally validated in preliminary trials (as described in section~\ref{sec:pe_d}).
	    
	    Based on the preliminary results detailed in section~\ref{sec:pe_r} we have decided to use composite (kernel \textit{and} cluster based) scrambling for our experiment.
	    The scrambling kernel was constant at \SI{6}{\pixel} (following considerations explained in sections~\ref{sec:pe_d_rt}) 
	    and “hard” recognition trials had a scrambling cluster size of \SI{10}{\pixel} while “easy” trials had a scrambling cluster of \SI{22}{\pixel}.
	    The quality of this baseline is assessed under section~\ref{sec:pe_d_fm}.
	    
	    Scrambling was done via Pyxrand \citep{pyxrand}, a Python script written for the purpose of this thesis and openly published on GitHub.
	    The script provides both cluster-based and kernel-based scrambling:
	    \subsubsection{Cluster-Based Scrambling}
		The Cluster-Based Scrambling functionality recognises the face region of interest (ROI) by scanning for pixel lines with few unique values, and then divides the face ROI in square clusters of predefined sizes.
		The clusters then get permuted and rewritten in-place on the image - this is done via the \colorbox{vlg}{\texttt{montage2d}} function (of the \colorbox{vlg}{\texttt{scikits\_image}} python package), which  was adapted for this purpose and contributed to the package for the benefit of the scientific community as part of this thesis.
		Finally, the script fills the image background (outside of the ROI) with homogeneous values.
	    \subsubsection{Kernel-Based Scrambling}\label{sec:m_vs_si_kbs}
		The Kernel-Bases scrambling variant is performed by remapping every pixel of the image via the \colorbox{vlg}{\texttt{geometric\_transform}} function provided by the SciPy ecosystem \citep{scipy,Oliphant2007}.
		New positions are computed via a function which adds a random integer in the $[-K;K]$ ($K$ being the scrambling kernel integer) interval to both the X and Y coordinates of the said pixel.
		Effectively, this redistributes each pixel in an area of $[-K;K]$ around its original position with a standard deviation ($\sigma$) of $\approx 0.96K$ (as calculated in figure~\ref{eq:lrgn}).
    \section{Functional MRI}\label{sec:m_fmri}
	Patterns of neuronal activity are to be extrapolated from functional magnetic resonance imaging (fMRI) data.
	This data was acquired over a \SI{3}{\tesla} Siemens “TIM Trio” MRI scanner at the Central Institute of Mental Health in Mannheim.
	\subsection{Data Acquisition}
	    Anatomical images were acquired in the sagital plane via a T1-MPRAGE sequence \citep{Brant-Zawadzki1992}, with a reprtition time (TR) of \SI{1.57}{\second}, an echo time of \SI{2.7}{\milli\second}, and a flip angle of \SI{15}{\degree}.
	    The slice thickness and the X and Y-axis resolution (relative to the acquisition plane) of the tomographic data were \SI{1}{\milli\metre} (\SI{1}{\milli\metre$^3$} voxel size).
	    
	    Functional data was acquired via T2* weighted echo planar imaging (T2* EPI), with a repetition time (TR) of \SI{2}{\second}, an echo time of \SI{30}{\milli\second}, and a flip angle of \SI{80}{\degree}.
	    A total of 33 tomographic images were acquired in descending order, at a slice thickness of \SI{3}{\milli\metre} and an X and Y-axis resolution of the same dimensions (\SI{27}{\milli\metre$^3$} voxel size).
	\subsection{Data Processing}
	    The resulting fMRI data in \colorbox{vlg}{\texttt{.dcm}} (DICOM) format was converted to the more convenient \colorbox{vlg}{\texttt{.nii}} (NIfTI) via the MRIconvert \citep{MRIconvert} software package.
	    Further preprocessing steps have been done in accordance with the established guidelines of the group, and are documented (in simplified form) as part of the faceOM script suite \citep{faceOM} - which was written for the purpose of this thesis and openly published for the benefit of researchers pursuing similar analysis.
	    
	    \begin{wrapfigure}{r}{0.55\textwidth}
		\centering
		%~ \vspace{-1.2cm}
		\scalebox{0.31}{\includegraphics{./img/LC_pic.png}}
		\caption{Mask for the LC based on MNI coordinates extracted from in-vivo MRI localization \citep{Keren2009}. The mask was smoothed and the dynamic range remapped to [-1;1].}
		\label{fig:m_fmri_dp}
		\vspace{-1.2cm}
	    \end{wrapfigure}
	    
	    Data was analysed by a first-level computation of per-participant contrasts, and a subsequent second-level statistical analysis of these contrasts, resulting in a probability-of-activation map.
	    The second-level (significance determining) analysis step was done both with and without a locus coeruleus mask.
	    We used a mask drawn for the purpose of this thesis in the MARINA program \citep{Walter2003} and based on in-vivo magnetic resonance localization of the LC available in peer-reviewed literature \citep{Keren2009}.
	    Three representative cross sections of our mask are presented in figure~\ref{fig:m_fmri_dp}
		
	    Additionally, pupil diameter time course data obtained from infra-red pupillometry (see section~\ref{sec:m_om_pm}) was down-sampled to a rate of \SI{0.5}{\hertz} (to match the functional TR) and used as a signal regressor.
	    As it is yet unclear what temporal relationship the pupil dilation response and LC activation have in humans, we implement the regressor in a series of different variations:
	    \begin{itemize}
		\item Raw pupil dilation time course
		\item Pupil dilation time course convolved with the hemodynamic response function (HRF)
		\item Numerical first differential of the pupil dilation time course
		\item Numerical first differential convolved with the hemodynamic response function (HRF)
	    \end{itemize}
	    
	    All the above detailed NIfTI file processing and analysis was performed via SPM8 - a MATLAB\textsuperscript{\small\textregistered}-based brain imaging analysis suite.
    \section{Occulometry}\label{sec:m_om}
	As part of our experiments we have measured occulometric parameters of our participants during behavioural trials performed inside the MRI scanner.
	Occulometric data was captured via an infra-red SemsoMotoric Instruments (SMI) “IVIEW X\textsuperscript{\small\texttrademark} MRI-LR” \SI{60}{\hertz} fMRI-compatible eye tracker.
	\subsection{Eye Tracking}\label{sec:m_om_et}
	    More accurately, we are performing “gaze tracking” rather than “eye tracking”, as the system measures the gaze coordinates rather than eye-in-head angles.
	    We use “eye tracking” as a general term for both measurements, keeping with the trend in both industry specifications \citep{Bojko2006} as well as peer-reviewed publications \citep{Kirk2013}.
	    
	    Gaze coordinates are calculated by the firmware provided with our eye tracker, based on the position of the corneal light source reflection relative to the pupil.
	    As our stimulus presentation does not yet incorporate a drift correction mechanism, our eye-tracking data is subject to gross movement artefacts.
	    
	    The data thus obtained will not - or only tentatively - be described in the “Results” section.
	\subsection{Pupillometry}\label{sec:m_om_pm}
	    From the data provided by the aforementioned hardware we are separately extracting the pupil diameter.
	    The pupil diameter is determined by the firmware of our eye tracker via an edge detection algorithm with a perimeter-to-area ratio constraint
	    (keeping the detected pupil area as close as possible to a circular form).
	    The light source needed to afford signal for the eye-tracking camera is built-in and shines infra-red light.
	    This avoids triggering the pupillary light reflex \citep{Ellis1981} and distorting the pupil diameter data.
	    
	    Our data is slightly distorted due to a number of factors - most prominently subjects' eyelids covering parts of the pupils and dark make-up of some female participants being registered as pupil area.
	    These distortions are not critical for gaze point calculation (as long as they remain constant over the course of the measurement); 
	    but they do interfere non-linearly with the firmware-calculated pupil area 
	    (pupil area lost to eyelid coverage, for instance, scales quite complexly with overlap distance - see figure~\ref{eq:cs}).
	    To avoid such complex data-reconstruction algorithms we are considering using the pupil horizontal diameter as a reference measure for the pupil dilation response in lieu of the actual area.
	    These considerations are further addressed based on experimental results under section~\ref{sec:r_p}.
	    
	    As discussed in section~\ref{sec:r_p} we are also considering a number of other data post-processing methods (such as frequency filtering) to eliminate further measurement artefacts from our pupil diameter time course. 
    \section{Statistical Analysis}\label{sec:m_sa}
	For the analysis of our data we have implemented a number of statistic methods from different software packages.
	Our choice of methods took into account transparency and reproducibility, and we have thus - whenever possible - avoided closed-source and/or graphical user interface based software (such as for example MATLAB\textsuperscript{\small\textregistered} or SPSS\textsuperscript{\small\textregistered}).
	Many of the statistical analysis functions are called at compile time from the \LaTeX\ code of this document, meaning that many of our statistical results are live.
	
	For ease of overview we have compiled a short index with explicit naming and descriptions of our statistical analysis tools.
	The methods detailed hereafter are referred to in further sections of this document simply by their subsection names.
	\subsection{Paired T-Test}\label{sec:m_sa_pt}
	    The t-test is a method used to test the null hypothesis that the population means of two compared sample groups are equal.
	    The most frequently reported resulting measure of the test (the \textit{p}-value) represents the probability of the presented data being observed if the null hypothesis is true. 
	    The t-test makes a small series of assumptions, among which is the independence of data points and the normal distribution of data.
	    
	    Our experiments mainly rely on testing the same population on various conditions.
	    To adapt the t-test to this constraint (where data points are no longer independent) we use the paired sample t-test (also known as a related sample t-tests).
	    We implement this statistical method via the \colorbox{vlg}{\texttt{scipy.stats.mstats.ttest\_rel}} function of the SciPy ecosystem \citep{scipy,Oliphant2007}.
	    
	    Unless otherwise specified, in order to avoid inflating our significance we are testing per-participant means instead of raw data points.
	    This also affords the advantage of averting undocumented violations of the normal distribution assumption of the t-test -
	    since means of even non-normally distributed populations tend to be normally distributed.  
	\subsection{Standard Error of the Mean}\label{sec:m_sa_se}
	    Confusion around appropriate usage of the standard deviation (SD) and standard error of the mean (SEM) is a significant quality deficit and cause of imprecision in modern research \citep{Nagele2003}.
	    SD is a measure of single data point variability and tends to be constant over increasing sample size, 
	    whereas SEM is a measure of the data sample mean reliability and tends to decrease as the sample increases \citep{Altman2005,Streiner1996}.
	    
	    As we are in more cases concerned with per-category means than data point variability, our error bars of choice represent the SEM by default.
	    Cases in which graphical depiction of the SD is helpful in understanding the data at hand are described accordingly and explicitly.
	    
	    In our purpose-built scripts, faceRT \citep{faceRT} and faceOM \citep{faceOM}, we use the \colorbox{vlg}{\texttt{scipy.stats.sem}} and \colorbox{vlg}{\texttt{numpy.std}} functions of the SciPy ecosystem \citep{scipy,Oliphant2007} -
	    for the standard error of the mean and standard deviation respectively.
	\subsection{ANOVA}\label{sec:m_sa_a}
	    ANOVA, the so-called analysis of variance, is a very popular set of statistical methods in psychology and related fields.
	    Formally, ANOVA tests the null hypothesis that the population means of multiple groups are equal.
	    
	    The method, however, has a series of limitations:
	    \begin{itemize}
		\item ANOVA provides no information as to the group whose mean is outside the range of other groups, nor does it indicate whether this is one group or a number of groups.
		\item ANOVA does not deal well with missing measurements (which motivates scientists to perform ANOVA on means rather than on raw data points - and thus discard valuable information) \citep{Gueorguieva2004}.
		\item The assumptions made by ANOVA - independence of observations, normal distribution of residuals, and homoscedasticity \citep{Anderson1996} - are likely to be violated due to autocorrelation whenever using raw data points. 
	    \end{itemize}
	    Overall, statisticians involved in biology and psychology recommend against ANOVA and point to mixed models (see section~\ref{sec:m_sa_lm}) as a substitute \citep{Baayen2008,Gueorguieva2004}. 
	    In light of these recommendations are only using ANOVA in cases where raw data point sets represent per-participant or binned values 
	    (such as in questionnaires, or in error rates analysis - as seen under section~\ref{sec:pe_r_ss}).
	    
	    In those cases where we do choose to resort to ANOVA, we have to consider that the standard one-way ANOVA assumes all data points are independent.
	    For our applications - where we are testing the same sample of participants over a number of conditions - the appropriate variation of ANOVA would be the repeated measure ANOVA \citep{Gueorguieva2004}.
	    Our ANOVA function of choice is \colorbox{vlg}{\texttt{stats::aov}} \citep{Chambers1992} from the R statistical environment \citep{R}.
	    We are using this function in instances where we compute ANOVA results \textit{live} whenever this document is compiled; for this purpose we are using the interstats \citep{interstats} software package as a wrapper.
	    
	    Additionally, are using a pseudo-ANOVA, the “analysis of deviance” \citep{Hastie1992} to test the quality of our linear models. 
	\subsection{Mixed Models}\label{sec:m_sa_lm}
	    Current literature \citep{Baayen2008} recommends mixed models over ANOVA's standard linear model in order to describe the effect of multiple experimental conditions on measurements.
	    
	    We fit mixed models to our data via the \colorbox{vlg}{\texttt{lme4::lmer}} \citep{Bates2005,Bates2007} R function.
	    This function - though one of the most powerful current tools for mixed modelling - fails to provide \textit{p}-values.
	    The author has justified this \citep{Bates2006}, and generally recommended against using \textit{p}-values with such models.
	    In spite of this recommendation, we hold that \textit{p}-values are a stringent standard in quantifying the reliability of scientific observations -
	    and are using the alternate \colorbox{vlg}{\texttt{nlme::lme}} \citep{Pinheiro2013} function whenever we believe \textit{p}-values are needed.
	    
	    We are using these functions in many instances to compute mixed models \textit{live} in this document.
	    For this purpose we are using the interstats \citep{interstats} software package as a wrapper.
	    This software was written for the purpose of this thesis and openly published on GitHub \citep{github}.
	\subsection{Pearson's \textit{r}}
	    We use “Pearson's \textit{r}” for the quantitative description of variable dependence - in cases where we believe this dependence can be reasonably approximated by linear correlation.
	    Pearson's \textit{r} is a century-old method with widespread use in the empirical sciences \citep{Stigler1989}.
	    It is however noteworthy that it is only meaningful for linear correlations (or nonlinear correlations with linear components).
	    
	    To test for this statistical measure we are using the \colorbox{vlg}{\texttt{scipy.stats.pearsonr}} function from the Scipy ecosystem \citep{scipy,Oliphant2007}.
    \section{Questionnaires}
	We have decided to complement our studies with a series of validated and widely used questionnaires.
	These tests could provide valuable insight into the interplay between psychotypes on one side and behavioural parameters or their neurophysiological underpinnings on the other.
	For test presentation we have decided on a web-based approach, for which we have used a web-enabled version of the free and open source (FOSS) software testMaker \citep{testmaker}.
	
	The choice of tests was based on both general relevance to the host group's research, and specific relevance to emotional perception.
	According to these criteria, we have compiled a selection of 8 questionnaires:
	\begin{itemize}
	    \item The Autism Spectrum Quotient questionnaire (\textbf{AQ}) \citep{Baron-Cohen2001}
	    \item The World Health Organization Adult ADHD Self-Report Scale (\textbf{ASRS}) \citep{Kessler2005}
	    \item The 1996 revised Beck Depression Inventory (\textbf{BDI2}) \citep{Beck1996}
	    \item The short version of the Borderline Symptoms List (\textbf{BSL23}) \citep{Bohus2009}
	    \item The Empathizing Quotient questionnaire (\textbf{EQ}) \citep{Baron-Cohen2004}
	    \item The Systemizing Quotient questionnaire (\textbf{SQ}) \citep{Baron-Cohen2003a}
	    \item The Emotion Regulation Questionnaire (\textbf{ERQ}) \citep{Gross2003}
	    \item The Schizotypal Personality Questionnaire (\textbf{SPQ}) \citep{Raine1991}
	\end{itemize}
	The questionnaires were administered in German language, and translated and validated versions were provided by the test databases of the Central Institute of Mental Health.
	Translation references are as follows:
	\begin{itemize}
	    \item \textbf{AQ} - Translated by C.M. Freitag, Homburg
	    \item \textbf{ASRS} - Translated by Dr. P. Kirsch, JLU Gießen 2005
	    \item \textbf{BDI2} - Translated by Pearson Assessment \& Information GmbH (Frankfurt/M. 2010)
	    \item \textbf{BSL23} - Translated at the Department of Psychosomatic Medicine and Psychotherapy, Central Institute of Mental Health.
	    \item \textbf{EQ} - Translated by Dipl.-Psych. Jörn de Haen
	    \item \textbf{SQ} - Translated by Dipl.-Psych. Jörn de Haen
	    \item \textbf{ERQ} - Translated by Brigit Abler and Henrik Kessler \citep{Abler2009}
	    \item \textbf{SPQ} - Tranlsated by Klein, Andersen, and Jahn \citep{Klein1997}
	\end{itemize}
	The questionnaires were evaluated with questioPy\citep{questiopy}, a Python script written for the purpose of this thesis and openly published on GitHub\citep{github}.
    \section{Genetic Material Samples}
	We have harvested mouth epithelial cells from our participants for prospective genotyping.
	The cells were obtained as a saliva sample collected via ???.
	
	The resulting suspensions are stored at room temperature and shall be used whenever the participant number reaches an adequate size for genetic testing (~\SI{50}{participants}) 
	The choice to resort to saliva rather than whole-blood lymphoblast testing was made in light of cost and safety considerations, though we are aware of the method's limited reliability \citep{Philibert2008}.

\chapter{Results}                                                                          
    \section{Preliminary Experiments}\label{sec:r_pe}
	In order to establish an experimental paradigm which affords the comparison between emotion and pattern recognition, we need to select stimuli whose matching is correspondingly difficult.
	For the emotion recognition trials, we decided on faces with emotional concentrations of 40\% and 100\% (as discussed in section~\ref{sec:m_vs_ef}).
	
	In the following experiments we examine reaction times and error rates for the recognition of progressively scrambled images in comparison to the recognition of our two emotional concentrations.
	\py[pe1]{fig_pe1}
	
	The reaction time figures which we present include both per-participant and population sample bar plots (the latter marked as “ALL”).
	As our sample sizes are reduced, we acknowledge the limited reliability with which we can make claims pertaining to the general population. 
	
	We would, however, also like to indicate that both inter-trial and inter-participant variability is high (as seen in figure~\ref{fig:r_pe1}).
	The accuracy of the determined mean (reflected by the standard errors in figure~\ref{fig:r_pe1}) could be improved by more extensive testing - though the usefulness of doing so with respect to our search for an optimal baseline would be limited.
	Regardless of the quality of our mean, reaction times on individual trials would still often coincide even for less stringent or sub-optimal choices of a baseline.  
	This would happen on the trial level, as well as on the \textit{mean} level for certain participants.
	Please see the standard deviations in figure~\ref{fig:r_pe1} for a graphical representation of where approximately \SI{68}{\percent} of data points reside for each condition.
	\subsection{Simple Cluster-Based Scrambling}\label{sec:r_pe_ss}
	    For the results presented in figure~\ref{fig:r_pe_ss1} we have used a simple cluster-based scrambling algorithm and a hand-written balanced stimulus sequence 
	    (both detailed in section~\ref{sec:m_pe_ss}).
	    \py[pe_ss1]{fig_pe_ss1}
	    
	    Figure~\ref{fig:r_pe_ss1} confirms our assumption that reaction times for the difficult emotion recognition condition are significantly different from (one-tailed related sample t-test \textit{p}-value of 
	    \py[pe_ss1]{tex_nr(ttest_rel(data_pe_ss1[(data_pe_ss1['scrambling'] == 0) & (data_pe_ss1['intensity'] == 40)]['RT'], data_pe_ss1[(data_pe_ss1['scrambling'] == 0) & (data_pe_ss1['intensity'] == 100)]['RT'])[1]/2)}),
	    and higher than for the easy condition.
	    The results also support our assumption of the presence of a downward trend in reaction times over increasing scrambling cluster sizes.
	    Additionally we see the emergence of a plateau-phase over higher scrambling cluster sizes.
	    
	    To test the validity of this tentative interpretation, we have fitted a mixed model to the reaction times of pattern recognition trials.
	    The model presented in table~\ref{tab:r_pe_ss1} supports our interpretation by showing that:
	    \begin{itemize}
		\item There is a negative progression among the first three factors, with all factor values lying outside of the confidence intervals of all others.
		\item For the subsequent 4 factors there is no obvious trend, with all factor values lying within each other's confidence intervals. 
	    \end{itemize}
	    \py[pe_ss1]{lm(data=data_pe_ss1[(data_pe_ss1['scrambling'] != 0)], fixed='RT~COI', random='ID', title='Mixed model fitted to the raw data depicted in figure~\\ref{fig:r_pe_ss1}. Our factors are the conditions of interest (COI), excluding the emotion recognition trials. The mod el chooses the first factor (scrambling-06) as the base intercept. Note the downwards trend for the first 3 factors and that the last 4 factors\' values lie within each other\'s confidence intervals.', label='tab:r_pe_ss1')}

	    This plateau suggests that scrambling cluster sizes of \SI{14}{\pixel}, \SI{18}{\pixel}, \SI{22}{\pixel}, and \SI{26}{\pixel} possessed the same recognition-relevant quality, which the \SI{6}{\pixel} scrambling cluster - at least - did not.
	    If this purported feature is to explain the reaction time distribution, it should emerge at a scrambling cluster size of around \SI{10}{\pixel}.
	    Visual scrutiny of the images suggested that the emergent feature may be recognizable facial elements - more precisely the eyes (with the iris having an outer diameter of approximately \SI{8}{\pixel}).
	    To test our hypothesis and define a more appropriate baseline for simple visual recognition, we devised a new set of experiments - described in section~\ref{sec:m_pe_cs}, and analyzed in section \ref{sec:r_pe_cs}.
	    
	    Though the images used in this first experiment run shall not be used as the definitive baseline, we conduct a baseline search to use for our further analysis in this section.
	    As we can see in table~\ref{tab:r_pe_ss2} pattern recognition of figures produced with a \SI{6}{\pixel} scrambling cluster recommends itself as the least unlikely (to keep with accurate terminology of the hypothesis testing we are performing) condition to share an equivalent baseline with difficult emotion recognition.
	    Similarly, it is pattern recognition of images processed with a \SI{22}{\pixel} scrambling cluster which we choose as our best-guess reaction time equivalent for easy emotion recognition.
	    \py[pe_ss1]{p_table(data_pe_ss1, ['COI', 'emotion-easy', 'emotion-hard'], ['scrambling', 6, 10, 14, 18, 22, 26], refcap='Proposed best estimators (scrambling cluster sizes in \\SI{}{px})', caption='Two-tailed paired sample t-test \\textit{p}-values for the data in figure~\\ref{fig:r_pe_ss1}; testing the probability of our measured observations if the compared conditions share the same mean. Note that - though unorthodox - this test does not lose accuracy due to multiple comparisons. As we are looking for the group least likely to reject the null hypothesis, multiple comparison actually makes the test more stringent.', label='tab:r_pe_ss2')}
	    
	    In addition to the reaction time per-category analysis, we attempted to gauge the maximal window for reaction times which we should provide in further experiments.
	    For this analysis we have selectively looked at reaction times for images processed with \SI{0}{\pixel}, \SI{6}{\pixel}, and \SI{22}{\pixel} scrambling clusters (due to considerations detailed above and documented in table~\ref{tab:r_pe_ss2}).
	    \py{fig_pe_ss2}
	    
	    Figure~\ref{fig:r_pe_ss2} shows the reaction time distribution for the said subset of trials.
	    The mean value ($\mu$) for the reaction times is approximately \pySI{np.around(data_pe_ss2['RT'].mean(), decimals=2)}{\second}, 
	    and the standard deviation ($\sigma$) of the fitted normal distribution is approximately \pySI{np.around(np.std(data_pe_ss2['RT']), decimals=2)}{\second}.
	    Based on the three-sigma rule (see figure~\ref{eq:3s} for a general expression) we have decided that a reaction time cut-off of \pySI{np.around(data_pe_ss2['RT'].mean()+3*np.std(data_pe_ss2['RT']), decimals=2)}{\second} should not statistically distort our data.
	    
	    We have also tried to analyse the error rates in our Hariri-style matching task in order to extract more information on the appropriateness of our proposed baselines.
	    The results herefor are presented in figure~\ref{fig:r_pe_ss3}.
	    \py[pe_ss3]{fig_pe_ss3}

	    Our error rate distribution over categories proves difficult to interpret.
	    It is obvious for one thing that error rates for most categories vary greatly among participants.
	    It also seems possible that emotion recognition in easy trials may be more robust than image matching in all other categories (owing to the constant null rate).
	    \py[pe_ss3]{av(data_pe_ss3, 'ER~COI+Error(ID)', title='One-way repeated measure ANOVA table for the data depicted in figure~\\ref{fig:r_pe_ss3}. The $Pr(>F)$ value indicates the adjusted probability of observed results occurring by chance if the means of our defined categories are equal.', label='tab:r_pe_ss3')}
	    
	    To test this assumption we have performed an ANOVA (see table~\ref{tab:r_pe_ss3}).
	    The results of this analysis prove equally difficult to interpret. While the $Pr(>F)$ (nominally, \textit{p}) value lies above the popular significance threshold of \textit{p}<0.05, 
	    it is still below the more lax (though not unheard of) \textit{p}<0.1 threshold.
	    
	    To further document this analysis we have fitted a mixed model with computed \textit{p}-values (see table~\ref{tab:r_pe_ss3a}).
	    This model indicates that, while it could find 3 factor values which were significantly different than the base intercept, it found 4 which were not.
	    Note that it is not abnormal for single comparison to find significant effects though ANOVA does not (more on this in section~\ref{sec:m_sa}).
	    We take this as supporting our decision no not reject the null hypothesis of all categories sharing the same mean.
	    \py[pe_ss3]{lm(data_pe_ss3, fixed='ER~COI', random='ID', title='Mixed model fitted to the raw data presented in figure~\\ref{fig:r_pe_ss3}. The model chooses easy emotion recognition (incidentally our category suspected of deviating from the mean) as the base intercept. The confidence interval for the factor values is annotated in the parentheses.', label='tab:r_pe_ss3a', model='nlme')}
	\subsection{Composite (Kernel-and-Cluster-Based) Scrambling}\label{sec:r_pe_cs}
	    Based on the results detailed in section ~\ref{sec:r_pe_ss} we have performed additional experiments using a series of improved method choices (detailed in section \ref{sec:m_pe_cs}).
	    
	    We have re-run experiments with kernel-and-cluster scrambled images a number of times.
	    This was based on progressively noticing parameters in need of adjustment so as to create a most faithful emulation of the in-scanner visual experience.
	    For the sake of brevity, we will only discuss the results of our final and most faithful emulation here in this section.
	    Data from the previous runs is appended for the sake of completion under section~\ref{sec:sm_ndper}.
	     
	    \py[pe_cs3]{fig_pe_cs3} 
	    
	    Figure~\ref{fig:r_pe_cs3} presents the the data from the final preliminary experiment run - which best approximates the visual input participants will be receiving in the fMRI trials, and which we trust above all others as an appropriate guideline for further experimental choices.
	    An obvious feature here is the disappearance of the plateau phase, which supports our hypothesis regarding its emergence due to facial features with a horizontal and/or vertical spatial frequency of around \SI{8}{\pixel} (formulated in section~\ref{sec:r_pe_ss}; this is only partly corroborated by other test runs, as seen in supplementary figures~\ref{fig:r_pe_cs1} and~\ref{fig:r_pe_cs2}).
	    
	    We have tried to determine the best-guess equivalent for our categories of interest - in terms of reaction times - using a t-test \textit{p}-value listing (see table~\ref{tab:r_pe_cs3}).
	    Our results indicate that pattern recognition of figures produced with a \SI{11}{\pixel} scrambling cluster is the least unlikely condition to share an equivalent baseline with difficult emotion recognition.
	    Similarly, it is pattern recognition of images processed with a \SI{23}{\pixel} scrambling cluster which we choose as our best-guess reaction time equivalent for easy emotion recognition.
	    \py[pe_cs3]{p_table(data_pe_cs3, ['COI', 'emotion-easy', 'emotion-hard'], ['scrambling', 11, 15, 19, 23, 27], refcap='Proposed best estimators (scrambling cluster sizes in \\SI{}{px})', caption='Table of two-tailed paired sample t-test \\textit{p}-values for the data in figure ~\\ref{fig:r_pe_cs3}, testing the probability of our observations if the compared conditions share the same mean. Note that - though unorthodox - this test does not lose accuracy due to multiple comparisons. As we are looking for the group least likely to reject the null hypothesis, multiple comparison actually makes the test more stringent.', label='tab:r_pe_cs3')}
	    
	    It should be noted that though not significantly different, both of these simple visual recognition reaction times undershoot their counterpart emotion recognition reaction times.
	    In light of this we have decided to use a \SI{6}{\pixel} kernel- and \SI{10}{\pixel} cluster-scrambled image baseline for difficult emotion recognition, 
	    and a \SI{6}{\pixel} kernel- and \SI{22}{\pixel} cluster-scrambled image baseline for easy emotion recognition.
	    
	    \py[pe_cs5]{fig_pe_cs5}
	    
	    To re-examine potential weaknesses of our baseline concept (related to error rate distribution and further discussed in section~\ref{sec:r_pe_ss}), we have also analysed error rates for this last and methodologically most faithful preliminary experiment run.
	    Figure~\ref{fig:r_pe_cs5} presents a slightly divergent picture from what we have seen in figure~\ref{fig:r_pe_ss3}.
	    For one thing, the easy emotion recognition condition no longer has a null error rate.
	    This time, however, there is a significant error rate divergence for one of our conditions compared to all of the others (as seen in figure~\ref{fig:r_pe_cs5}, and verified in table~\ref{tab:r_pe_cs5}).
	    \py[pe_cs5]{av(data_pe_cs5, 'ER~COI+Error(ID)', title='One-way repeated measure ANOVA table for the data depicted in figure~\\ref{fig:r_pe_cs5}. The $Pr(>F)$ value indicates the adjusted probability of observed results occurring by chance if the means of our defined categories are equal.', label='tab:r_pe_cs5')}
	
	    While this finding confirms limited reliability for all of our our preliminary trials, resulting doubts that our baseline may be unsuited due to differing error rates should also be treated with according scepticism.
	    
	    We also analyse whether our slightly over-cropped reaction time window seems to distort our data.
	    In section~\ref{sec:m_pe_cs} we have detailed our decision to settle for a \SI{4}{\second} reaction time window, the predicted ideal cropping being \pySI{np.around(data_pe_ss2['RT'].mean()+3*np.std(data_pe_ss2['RT']), decimals=2)}{\second}, as calculated in section~\ref{sec:r_pe_ss}.
	    In figure~\ref{fig:r_pe_cs4} we see that reaction times seem to fade approximately \SI{0.5}{\second} before our cut-off, and the fitted normal distribution does not indicate expected results beyond that point.
	    We take these results as confirmation that our selected cut-off time is not over-stringent for the selected conditions (easy and hard emotion recognition, plus visual recognition of scrambled images processed with a \SI{6}{\pixel} scrambling kernel and \SI{11}{\pixel} and \SI{23}{\pixel} scrambling cluster respectively).	    
	    \py[pe_cs4]{fig_pe_cs4}
	    
	    Reproducibility issues apparent in the data presented in this section - along with broader conclusions to be drawn from these experiments - are further discussed in section \ref{sec:d_pe}.
	    
	\subsection{Final Model Quality Assessment}
	    We have decided to fit a mixed model to both the reaction times and error rates of our last-run data in order to check the quality of the baseline we have proposed as the best-guess solution based on the data detailed above.
	    For this model we are defining “scrambling” and “difficulty” as our factors of interest (independent variables).
	    Easy emotion recognition and its corresponding pattern recognition variant (\SI{6}{\pixel} kernel- and \SI{23}{\pixel} cluster-scrambled) are defined as easy,
	    while hard emotion recognition and its corresponding pattern recognition variant (\SI{6}{\pixel} kernel- and \SI{11}{\pixel} cluster-scrambled) are defined as hard.
	    
	    For our ideal baseline we would be expecting a null value for the scrambling factor and a positive value for the difficulty factor in the reaction time model.
	    In the error rate model we would be expecting a null value for the scrambling factor and either a null or positive value for the difficulty factor.
	    \py[pe_cs5]{lm(data=data_pe_cs5[(data_pe_cs5['COI'] == 'scrambling-11') | (data_pe_cs5['COI'] == 'scrambling-23') | (data_pe_cs5['COI'] == 'emotion-hard') | (data_pe_cs5['COI'] == 'emotion-easy')], fixed='ER~difficulty*scrambled', random='ID', title='Mixed model fitted to the raw error rate data depicted in figure~\\ref{fig:r_pe_cs5}. Our factors are scrambling (yes or no) and difficulty (easy or hard). The model chooses "easy" and "not scrambled" as the intercept value. Difficulty and scrambling show no significant factor value, while their interaction does. The base intercept is not significant meaning that the base error rate of the model may not be be reliably considered to lie above zero.', label='tab:r_pe_cs5s')}
	    \py[pe_cs3]{lm(data=data_pe_cs3[(data_pe_cs3['COI'] == 'scrambling-11') | (data_pe_cs3['COI'] == 'scrambling-23') | (data_pe_cs3['COI'] == 'emotion-hard') | (data_pe_cs3['COI'] == 'emotion-easy')], fixed='RT~difficulty*scrambled', random='ID', title='Mixed model fitted to the raw reaction time data depicted in figure~\\ref{fig:r_pe_cs3}. Our factors are scrambling (yes or no) and difficulty (easy or hard). The model chooses "easy" and "not scrambled" as the intercept value. The base intercept is significant, indicating reaction times of the model are above zero. The difficulty factor is also positive and significant meaning that difficulty adds to response latency. All other factors are not significantly other than zero.', label='tab:r_pe_cs3s')}
	    
	    Tables~\ref{tab:r_pe_cs5s} and~\ref{tab:r_pe_cs3s}, provide a succinct and reliable validation for our proposed baseline model.
	    
	    Table~\ref{tab:r_pe_cs5s} recommends the choice of scrambling as a suitable baseline creation method for our task.
	    The factor determined for this dimension is null, meaning that scrambling does not compromise the behavioural equivalence of emotional testing and baseline trials.
	    Difficulty does also not seem to impact the error rate, which we consider fortunate, though a higher error rate for difficult trials would also have been acceptable.
	    The interaction effect of difficulty and scrambling is probably an effect of the deviating hard emotion recognition category we have observed under figure~\ref{fig:fig_pe_cs5}, with an inverted balance due to our model's choice of a base intercept.
	    
	    We would like to point out that the null residual variance of our model indicates that it could be overparameterized.
	    We have tried a series of alternate models (“scrambled” and “difficulty” - not including interaction -, “scrambled” alone and “difficulty” alone), but all resulted in null residual variance - 
	    see tables~\ref{tab:r_pe_cs5s1}, \ref{tab:r_pe_cs5s2}, and~\ref{tab:r_pe_cs5s3}.
	    
	    The reaction time model (table~\ref{tab:r_pe_cs3s}) paints a similar picture: with a significant value for the difficulty factor and no significance either for scrambling nor for scrambling and difficulty interaction.
	    We would conclude that in spite of our reduced sample sizes and aforementioned reproducibility issues our scrambling algorithm represents a viable type for a baseline (further discussion in section~\ref{sec:d_pe}.
    \vspace{0.2cm}
    \begin{center}
    \textbf{Henceforth we are dealing with multimodal data obtained in our optometry-fMRI set-up (described under sections~\ref{sec:m_fmri} and~\ref{sec:m_om}):}
    \end{center}
    
    \section{Participant Response Analysis}\label{sec:r_ra}
	We are testing parameters of the participant input in our main experiments similarly to how we have assessed our preliminary trials.
	This is to determine the extent to which we can rely on our baseline in in the further analysis of this data set.

	Due to equipment malfunction participant input data is only available for 8 of our 12 participants.
	\subsection{Reaction Times}\label{sec:r_ra_rt}
	    \py[ra_rt]{fig_ra_rt}
	    \py[ra_rt]{lm(data=data_ra_rt[(data_ra_rt['difficulty'] == 'easy') | (data_ra_rt['difficulty'] == 'hard')].reset_index(), fixed='RT~difficulty*scrambled', random='ID', title='Mixed model fitted to the raw reaction times depicted in figure~\\ref{fig:fig:ra_rt}. Our factors are scrambling (yes or no) and difficulty (easy or hard). The model chooses "easy" and "not scrambled" as the intercept value. Difficulty shows a significant positive effect, which is to be expected. Scrambling also shows a significant positive effect - meaning that pattern recognition trials have higher reaction times. This reaction time difference is also not a global offset as difficulty and scrambling interact to form a significant negative effect.', label='tab:r_ra_rt')}
	    The reaction times for our experiment are shown in figure~\ref{fig:ra_rt}.
	    An obvious feature (especially in comparison to the reaction time relationships we sought to find in section~\ref{sec:r_pe}) is the highly significant difference between easy emotion recognition and “easy” pattern recognition -
	    two tailed paired t-test \textit{p}-value of \py[ra_rt]{tex_nr(ttest_rel(data_ra_rt[(data_ra_rt['difficulty'] == "easy") & (data_ra_rt['emotion'] == "scrambled")].groupby(level=0)['RT'].mean(), data_ra_rt[(data_ra_rt['difficulty'] == "easy") & (data_ra_rt['emotion'] != "scrambled")].groupby(level=0)['RT'].mean())[1]/2)}.
	    
	    To further describe this potentially disruptive development for our baseline, we have fitted a mixed model with reaction time as our dependent variable, and emotion and difficulty as our factors (independent variables).
	    Overall, the model listed in table~\ref{tab:r_ra_rt} gives a quantitative description which accurately describes that pattern recognition trials prompted significantly higher reaction times.
	    Also, the emergence of a significant interaction factor in the model indicates that among the easy and hard pattern recognition trials, the difference in reaction times does not correspond directly to what the effect of the difficulty factor prescribes.
	    These observations on a whole prompt caution in further utilisation of our baseline and predict limited reliability of results thus obtained.
	\subsection{Error Rates}\label{sec:r_ra_er}
	    Additionally we have performed an error rate analysis to check for baseline weaknesses as seen in section~\ref{sec:r_pe_cs}.
	    \py[ra_er]{fig_ra_er}
	    \py[ra_er]{av(data_ra_er.reset_index(), 'ER~difficulty*scrambled+Error(ID)', title='One-way repeated measure ANOVA table for the data depicted in figure~\\ref{fig:ra_er}. The $Pr(>F)$ value indicates the adjusted probability observed results occurring by chance if the means of our defined categories are equal.', label='tab:r_ra_er')}
	    
	    Both figure~\ref{fig:ra_er} and the ANOVA test run on the portrayed data (under table~\ref{tab:r_ra_er}) show that the baseline is not robust in terms of error rates.
	    Note that the highly significant effect observed for the difficulty factor is not a weakness in the model per se - changes in error rates over difficulty alone would be acceptable and expected.
	    The significant interaction effect of difficulty and scrambling, however, indicates that the given difficulty effect on emotion recognition trials is not reproduced by pattern recognition trials, thus coming to the detriment of the pattern recognition baseline.	    
    
    \section{Pupillometry}\label{sec:r_p}
	\subsection{Methodological Reliability}\label{sec:r_p_mr}
	    Seeing as we are performing the debut pupillometric study on our present set-up, we have decided to perform at least a rough assessment of the quality of our measurements.
	    A number of possible inaccuracies became apparent during data acquisition (described in detail under section~\ref{sec:m_om_pm}) - 
	    and we are addressing these accordingly in the following paragraphs.
	    
	    To reiterate, it seemed concerning that Y-axis pupil diameter is often obfuscated by eyelid coverage.
	    Seeing as X-axis diameter incurred no such interference, we believe a simple analysis of correlation can show in how far our measurements are congruent and in how far the Y-axis measurements are reliable.
	    
	    Figure~\ref{fig:p_mr} shows raw data points and means for pupil diameters.
	    Due to the overwhelming amount of raw data points sharing the same variance, Pearson's \textit{r} is numerically undefined for our raw data.
	    We are thus fitting the correlation to the data point means. 
	    Here we see a Pearson's \textit{r} of \py[p_mr]{tex_nr(data_p_mr[0][0])} and a \textit{p}-value for non-correlation of approximately \py[p_mr]{tex_nr(data_p_mr[0][1])}.
	    It is safe to conclude from this result that our measurements are congruent.
	    We will thus be using the mean of X-axis and Y-axis diameters as our preferred diameter approximation.
	    
	    The observations we made during data acquisition pertaining to interference from the eyelids with Y-axis measurements are however also confirmed to be true.
	    A slight offshoot of raw data points is visible for higher values underneath the regression curve.
	    This very probably represents single measurements during which the eyelid partially covered the pupil.  
	    while this offshoot is obscured for mean values, its effect can be traced in the linear formula of the regression line: 
	    $y = $\py[p_mr]{tex_nr(data_p_mr[1])}$x$\py[p_mr]{tex_nr(data_p_mr[2])}.
	    The \textit{y}-intercept (\py[p_mr]{tex_nr(data_p_mr[2])}) indicates that there is a fixed offset between X-axis and Y-axis measurements, by which across our measurement range (approx $[20;75]$) Y-axis values are smaller than X-axis values.
	    \py[p_mr]{fig_p_mr}
\chapter{Discussion}
    \section{Preliminary Experiments}\label{sec:d_pe}
	Most conclusions drawn from our preliminary experiments (for instance in section~\ref{sec:m_pe_cs} or section~\ref{sec:r_pe_cs}) are only marginally reliable due to the high inter-run variability noticed throughout our re-runs in section~\ref{sec:r_pe}.
	We believe the chief reason for this is our reduced participant sample (n = 6 to 7), and we would recommend a more thorough examination on a larger population to better support any claims pertaining to accurate reaction time and error rate comparisons.
	
	Some hallmarks, however, have been constant over all our runs, and we believe these deserve to be reported with more pronounced certainty.
	\subsection{Reaction Times over Scrambling Cluster Sizes}
	    Our preliminary experiments show that even when confronted with very high variability in results there is a robust decay of reaction times as scrambling cluster sizes increase.
	    We report this is happening both with composite scrambling and with simple cluster-based scrambling.
	    
	    Though this effect was assumed by us a priori, and is hardly surprising, we hold that it validates our scrambling software for ample uses in Neuropsychology.
    \section{Pyxrand}
    A number of studies\cite{Rakover2013} have used scrambled face stimuli for their experiments.
    The stimuli are often prepared either by hand, or through time-consuming per-picture editing in an image manipulation program.
    Though automated scrambling software already exists \cite{Conway2008}, its scope is a lot narrower and its algorithms are nor released publicly.
    
    In findings 
    
    We recommend our software over possible alternatives due to its time-efficient batch functionality and its capacity to scramble both with and without specifically distorting facial features.
In addition to the faculty advisers mentioned in the preamble of this document, we would like to explicitly give thanks to a number of other members of the department of clinical psychology at the Central Institute of Mental Health in Mannheim.
Their expertise in designing and performing experiments, and analysing results was instrumental to the success of this thesis.
\begin{multicols}{2}
    \begin{itemize}
	\item Martin Gerchen
	\item Daniela Mier
	\item Carina Sauer
	\item Gabriela Stoessel
    \end{itemize}
\end{multicols}
\vspace{0.5cm}
We would also like to extend our gratitude to the numerous other people with whom we have interacted in the process of writing this thesis, and whose goodwill and effort made a significant difference pertaining to the quality of this work.
\begin{itemize}
    \item Øystein Bjørndal, of the Norwegian Defence Research Establishment
    \item Ben Bolker, of McMaster University
    \item Denis A. Engemann, of the Juelich Research Centre in Cologne
    \item Laurent Gautier, of the Technical University of Denmark
    \item Marek Hlavac, of Harvard University
    \item Philip Leifeld, of the University of Konstanz
    \item Christopher Louden, of the University of Texas
    \item Geoffrey M. Poore, of Union University
\end{itemize}

\chapter{Supplementary Material}
    \begin{figure}[H]
	\[ \sigma_{x,y} = \sqrt{\sigma_{x}^{2}+\sigma_{y}^{2}} = \sqrt{2\sigma_{x}^{2}} \approx \sqrt{2 \cdot (0.68 K)^{2}} \approx 0.96K\]
	\caption{Here we calculate the standard deviation ($\sigma$) of the distance from the original pixel location to the new pixel location following kernel-based scrambling, as detailed in section~\ref{sec:m_vs_si_kbs}. The standard deviation along one axis is defined as $0.68K$ (\SI{68}{\percent} of the scrambling kernel, $K$) following the 68–95–99.7 rule, though experimentally we have found slightly lower values ($\approx 0.6K$).}
	\label{eq:lrgn}
    \end{figure}
    \begin{figure}[H]
	\[\Pr(\mu - \sigma \le x \le \mu + \sigma) \approx 0.6827 \]
	\[\Pr(\mu - 2\sigma \le x \le \mu + 2\sigma) \approx 0.9545 \]
	\[\Pr(\mu - 3\sigma \le x \le \mu + 3\sigma) \approx 0.9973 \]
	\caption{The general expression of the three-sigma rule, in mathematical notation, for $x$ being an observation from a normally distributed random variable, $\mu$ being the mean of the distribution, and $\sigma$ being its standard deviation. The three-sigma rule (also known as the “68–95–99.7 rule”) states that nearly all values lie within 3 standard deviations of the mean in a normal distribution.}
	\label{eq:3s}
    \end{figure}
    \begin{figure}[H]
	\begin{eqnarray*}
	    A(h)&=&\frac{R^2}{2}\left[2\arccos\left(1-\frac{h}{R}\right) - \sin\left(2 \arccos\left(1-\frac{h}{R}\right)\right) \right]\\
	    &=&R^2 \left [\arccos{\left (1-\frac{h}{R}\right)} - \left (1-\frac{h}{R}\right) \sqrt{2 \frac{h}{R} - \frac{h^2}{R^2}} \right]
	\end{eqnarray*}
	\caption{The area of a circle segment $A$ defined relative to the height of that segment, $h$ - with $R$ as the circle radius. We use this as a function for how much of the pupil area will be lost relative to how far an eyelid covers it. For the sake of simplicity we approximate the eyelid with a straight rather than very slightly curved line.}
	\label{eq:cs}
    \end{figure}
    \section{Non-Definitive Preliminary Experiment Runs}\label{sec:sm_ndper}
	\py{fig_pe_cs1}
	\py{p_table(data_pe_cs1, ['COI', 'emotion-easy', 'emotion-hard'], ['scrambling', 11, 15, 19, 23, 27], refcap='Proposed best estimators (scrambling cluster sizes in \\SI{}{px})', caption='Table of two-tailed paired sample t-test \\textit{p}-values for the data in figure ~\\ref{fig:r_pe_cs1}, testing the probability of our observations if the compared conditions share the same mean. Note that - though unorthodox - this test does not lose accuracy due to multiple comparisons. As we are looking for the group least likely to reject the null hypothesis, multiple comparison actually makes the test more stringent.', label='tab:r_pe_cs1')}
	\py{fig_pe_cs2}
	\py{p_table(data_pe_cs2, ['COI', 'emotion-easy', 'emotion-hard'], ['scrambling', 6, 10, 14, 18, 22], refcap='Proposed best estimators (scrambling cluster sizes in \\SI{}{px})', caption='Table of two-tailed paired sample t-test \\textit{p}-values for the data in figure ~\\ref{fig:r_pe_cs2}, testing the probability of our observations if the compared conditions share the same mean. Note that - though unorthodox - this test does not lose accuracy due to multiple comparisons. As we are looking for the group least likely to reject the null hypothesis, multiple comparison actually makes the test more stringent.', label='tab:r_pe_cs2')}
	\py{lm(data=data_pe_cs2[(data_pe_cs2['scrambling'] != 0)], fixed='RT~COI', random='ID', title='Mixed model fitted to the raw data depicted in figure~\\ref{tab:r_pe_cs2a}. Our factors are the conditions of interest (COI), excluding the emotion recognition trials. The model chooses the first factor (scrambling-06) as the base intercept. Note the downwards trend for the first 3 factors and that the last 3 factors\' values land confidence intervals are almost identical.', label='tab:r_pe_cs2a')}
	\py[pe_cs5]{lm(data=data_pe_cs5[(data_pe_cs5['COI'] == 'scrambling-11') | (data_pe_cs5['COI'] == 'scrambling-23') | (data_pe_cs5['COI'] == 'emotion-hard') | (data_pe_cs5['COI'] == 'emotion-easy')], fixed='ER~difficulty+scrambled', random='ID', title='Linear model fitted to the raw error rate data depicted in figure~\\ref{fig:r_pe_cs5}. Our factors are scrambling (yes or no) and difficulty (easy or hard) - not including their interaction. The model chooses "easy" and "not scrambled" as the intercept value.', label='tab:r_pe_cs5s1')}
	\py[pe_cs5]{lm(data=data_pe_cs5[(data_pe_cs5['COI'] == 'scrambling-11') | (data_pe_cs5['COI'] == 'scrambling-23') | (data_pe_cs5['COI'] == 'emotion-hard') | (data_pe_cs5['COI'] == 'emotion-easy')], fixed='ER~difficulty', random='ID', title='Linear model fitted to the raw error rate data depicted in figure~\\ref{fig:r_pe_cs5}. Our factor is and difficulty (easy or hard). The model chooses "easy" as the intercept value', label='tab:r_pe_cs5s2')}
	\py[pe_cs5]{lm(data=data_pe_cs5[(data_pe_cs5['COI'] == 'scrambling-11') | (data_pe_cs5['COI'] == 'scrambling-23') | (data_pe_cs5['COI'] == 'emotion-hard') | (data_pe_cs5['COI'] == 'emotion-easy')], fixed='ER~scrambled', random='ID', title='Linear model fitted to the raw error rate data depicted in figure~\\ref{fig:r_pe_cs5}. Our factors are scrambling (yes or no). The model chooses "not scrambled" as the intercept value.', label='tab:r_pe_cs5s3')}
    

\footnotesize
\bibliography{./bib}
\bibliographystyle{plainurl}
\clearpage{}
\null
\vfill
\begin{center}
    \footnotesize\textcolor{mg}{This document is versioned via Git and the latest version is available in \colorbox{vlg}{\texttt{.tex}} format at\\ https://github.com/TheChymera/masterarbeit}
    
    \footnotesize\textcolor{mg}{This document is formatted for open publication review loosely based on the guidelines of \textbf{OPR} detailed at\\ \url{https://github.com/TheChymera/OPR.Concepts-Guidelines-for-OPR}}.
\end{center}
\end{document}

