\documentclass[a4paper,romanian,english]{book}
\usepackage[utf8]{inputenc}
\usepackage[T1]{fontenc}
\usepackage{babel}
\usepackage{graphicx}
\usepackage{caption}
\usepackage[super]{nth} % nice bindings for ordinal numerals: "\nth{7}"
\usepackage{subcaption}
\usepackage{multicol}
\usepackage{color}
\usepackage{tabularx} %Make tables with equally distributed ell space
\usepackage{float}
\usepackage[authoryear,square]{natbib}
\usepackage{wasysym}
\usepackage{verbatim} 
\usepackage[labelfont={bf}, margin=0.8cm]{caption}
\usepackage{fancyhdr}
\usepackage[iso, english]{isodate}
\usepackage{dashrule}
\usepackage{siunitx} %pretty measurement unit rendering
\usepackage{catchfilebetweentags} % used to import the REAMDE w/o the reST header
\usepackage{courier} % For prettier monospace fonts (used for code)
\usepackage{color,listings} % background color highlighting for code
\usepackage[parfill]{parskip} % blank lines instead of indents to separate paragraphs
\usepackage[hidelinks]{hyperref} % Makes URLs, DOIs, and page links work. w/o colored bracketing.
\usepackage[outer=1.7cm, inner=2.5cm, top=2.5cm, bottom=2cm]{geometry}% Makes margins narrower.
\usepackage[autoprint=false, gobble=auto]{pythontex} % create figures on-line directly from python!
\usepackage{pgf} %reads .pgf files for figures with text annotations scaling separately from the graphic
\usepackage{wrapfig} %allow text to wrap around certain figures

\newcommand{\pySI}[2]{\py{'\\SI{' + str(#1) + '}{#2}'}} % lets you use \pySI as a SI equivalent if you need to print pycode output
\newcommand{\dotbox}[2][.5\linewidth]{\hbox to #1 {\dotfill}\noindent#2}  % make dotted lines for signature/input fields

\newcolumntype{Y}{>{\centering\arraybackslash}X} % make Y type for tabularx which is like X (equal space distribution) but also centered
\renewcommand{\theequation}{\arabic{equation}}
\widowpenalty1000 % no widows
\clubpenalty1000 % no orphans

\definecolor{vlg}{gray}{0.90}
\definecolor{lg}{gray}{0.60}
\definecolor{mg}{gray}{0.40}
\DeclareSIUnit\pixel{px}

\setcounter{secnumdepth}{3} %turns on numbering for subsubsections 

\makeatletter
\renewcommand\chapter{
	\thispagestyle{headings}
	\if@openright\cleardoublepage\else\clearpage\fi\global\@topnum\z@
	\secdef\@chapter\@schapter
	}
\makeatother

\begin{pythontexcustomcode}[begin]{py}
import subprocess
import sys

from scipy.stats import ttest_rel, ttest_ind
pytex.add_dependencies('/home/chymera/src/interstats/interstats.py')
sys.path.append('/home/chymera/src/interstats/')
from interstats import lm, av, tex_nr, p_table

import matplotlib as mpl
mpl.use("pgf")
pgf_with_custom_preamble = {
    "font.family": "serif", # use serif/main font for text elements
    "text.usetex": True,    # use inline math for ticks
    "pgf.rcfonts": False,   # don't setup fonts from rc parameters
    "pgf.preamble": [
         r"\usepackage{units}",         # load additional packages
         r"\usepackage{metalogo}",
         r"\usepackage{unicode-math}",  # unicode math setup
         r"\setmathfont{xits-math.otf}",
         r"\setmainfont{DejaVu Serif}", # serif font via preamble
         ]
}
mpl.rcParams.update(pgf_with_custom_preamble)

from pylab import *

# Set the prefix used for figure labels
fig_label_prefix = 'fig'
# Track figure numbers to create unique auto-generated names
fig_count = 0

def save_fig(name='', legend=False, fig=None, ext='.pgf', fig_width=1, fig_height=1):
    '''
    Save the current figure (or `fig`) to file using `plt.save_fig()`.
    If called with no arguments, automatically generate a unique filename.
    Return the filename.
    '''
    # Get name (without extension) and extension
    if not name:
        global fig_count
        # Need underscores or other delimiters between `input_*` variables
        # to ensure uniqueness
        name = 'auto_fig_{}_{}_{}-{}'.format(pytex.input_family, pytex.input_session, pytex.input_restart, fig_count)
        fig_count += 1
    else:
        if len(name) > 4 and name[:-4] in ['.pgf', '.svg', '.png', '.jpg']:
            name, ext = name.rsplit('.', 1)

    # Get current figure if figure isn't specified
    if not fig:
        fig = gcf()
    fig.set_size_inches(fig_width,fig_height)
    fig.savefig(name + ext)
    fig.clf()
    return name

def latex_environment(name, content='', option=''):
    '''
    Simple helper function to write the `\begin...\end` LaTeX block.
    '''
    return '\\vspace{0.3cm}\n\\begin{%s}%s\n%s\n\\end{%s}' % (name, option, content, name)

def latex_figure(name=None, caption='', label='', width=0.8):
    ''''
    Auto wrap `name` in a LaTeX figure environment.
    Width is a fraction of `\textwidth`.
    '''
    if not name:
        name = save_fig()
    content = '\\centering\n'
    content += '\\makeatletter\\let\\input@path\\Ginput@path\\makeatother\n'
    content += '\\input{%s.pgf}\n' % name
    if not label:
        label = name
    if caption and not caption.rstrip().endswith('.'):
        caption += '.'
    # `\label` needs to be in `\caption` to avoid issues in some cases
    content += "\\caption{%s\\label{%s:%s}}\n" % (caption, fig_label_prefix, label)
    return latex_environment('figure', content, '[htp]')
\end{pythontexcustomcode}


\makeatletter
\def\maketitle{%
    \setlength{\parindent}{0cm}
    \null
    \pagestyle{empty}%
    \vfill
    \begin{center}\leavevmode
    \normalfont{}
    \vskip 1.5cm
    {\Large \textbf{\@title}\par}%
    \vskip 3cm
    {\Large Master Thesis\par}
    \vskip 0.5cm
    {\Large Presented to the Faculty of Biosciences\\at the Ruprecht-Karls-Universität Heidelberg\par}
    \vskip 5cm
    {\Large \@author\par}%
    {\Large \the\year\par}%
    \end{center}%
    \vfill
    \null
    \cleardoublepage{}
    \null
    \vfill{}
    \large
    This thesis was written at the \textbf{Central Institue of Mental Health} at the University of Heidelberg in the period of \textbf{2013-06-06} to \textbf{\today} under supervision of \textbf{Prof. Dr. Peter Kirsch}.
    \vskip 3cm
    1$^{st}$ \hspace{0.085cm}Appraiser: Prof. Dr. Rainer Spanagel \hfill Institute: Central Institute of Mental Health\\    
    2$^{nd}$ Appraiser: Prof. Dr. Daniel Durstewitz \hfill Institute: Central Institute of Mental Health\\
    \vskip 3cm
    I herewith declare that I wrote this Master thesis independently under supervision and used no other sources and aids than those indicated.
    \vskip 3.5cm
    \hdashrule{4cm}{1pt}{1pt 1pt}\hspace{5cm}\hdashrule{5cm}{1pt}{1pt 1pt}
    \\
    Date\hspace{8.2cm}Signature
    \vskip 2cm
    \hdashrule{\textwidth}{2pt}{1pt 1pt} 
    \vskip 2.5cm
    The thesis has to contain a summary in English.
    \vfill
    \null{}
    \setlength{\parindent}{0.55cm}
    \cleardoublepage
}

\makeatother
\title{“Neuronal Correlates of Occulometric Parameters in Face Recognition”}
\author{Horea-Ioan Ioanãş}

\linespread{1.45}
\begin{document}
\maketitle
\pagestyle{headings}
\setlength{\parindent}{0pt}
\setlength{\hangindent}{0pt}
\cleardoublepage
\thispagestyle{empty}
\chapter{Acknowledgements}
    In addition to the faculty advisers mentioned in the preamble of this document, we would like to explicitly give thanks to a number of other members of the department of clinical psychology at the Central Institute of Mental Health in Mannheim.
    Their expertise in designing and performing experiments, and analysing results was instrumental to the success of this thesis.
    \begin{multicols}{2}
	\begin{itemize}
	    \item Martin Gerchen
	    \item Daniela Mier
	    \item Carina Sauer
	    \item Gabriela Stoessel
	\end{itemize}
    \end{multicols}
    \vspace{0.5cm}
    We would also like to extend our gratitude to the numerous other people with whom we have interacted in the process of writing this thesis, and whose goodwill and effort made a significant difference pertaining to the quality of this work.
    \begin{itemize}
	\item Øystein Bjørndal, of the Norwegian Defence Research Establishment
	\item Ben Bolker, of McMaster University
        \item Denis A. Engemann, of the Juelich Research Centre in Cologne
	\item Laurent Gautier, of the Technical University of Denmark
	\item Marek Hlavac, of Harvard University
	\item Philip Leifeld, of the University of Konstanz
	\item Christopher Louden, of the University of Texas
	\item Geoffrey M. Poore, of Union University
    \end{itemize}

\setcounter{page}{1}
\tableofcontents{}
%~ \mbox{}
%START FIGURES AND DATA 
\begin{pycode}[pe_ss1]
pytex.add_dependencies('/home/chymera/src/faceRT/analysis/RTforCategories.py')
pytex.add_dependencies('/home/chymera/src/faceRT/analysis/data_functions.py')
pytex.add_dependencies('/home/chymera/src/faceRT/analysis/gen.cfg')
sys.path.append('/home/chymera/src/faceRT/analysis/')
import RTforCategories
data_pe_ss1 = RTforCategories.main(experiment='6px-4px-6steps', prepixelation=0, source='server', elinewidth=1, ecolor='0.3', make_tight={"pad": 0})
fig_pe_ss1 = latex_figure(save_fig(fig_width=6.64, fig_height=3), caption='Reaction times for Hariri-style face matching. Scrambled images for simple visual recognition were preprocessed only with a scrambling cluster of the sizes indicated in the graphic (sizes given in pixels). Reaction times for non-response trials were counted as \SI{5}{\second}. The error bars represent the standard deviation.', label='r_pe_ss1')
\end{pycode}
\begin{pycode}
pytex.add_dependencies('/home/chymera/src/faceRT/analysis/RTdistribution.py')
pytex.add_dependencies('/home/chymera/src/faceRT/analysis/data_functions.py')
sys.path.append('/home/chymera/src/faceRT/analysis/')
pytex.add_dependencies('/home/chymera/src/faceRT/analysis/gen.cfg')
import RTdistribution
keep_scrambling=(0, 6, 22)
data_pe_ss2 = RTdistribution.main(experiment='6px-4px-6steps', prepixelation=0, keep_scrambling=keep_scrambling, source='server', make_tight={"pad": 0}, print_title=False)
fig_pe_ss2 = latex_figure(save_fig(fig_width=6.64, fig_height=3), caption='Histogram of reaction times for Hariri-style face matching. Scrambled images for simple visual recognition were preprocessed only with a scrambling cluster. Missed targets have not been counted. For this graphic we have only indexed the '+ ', '.join([str(i) for i in keep_scrambling]) +' pixel scrambling cluster trials.', label='r_pe_ss2')
\end{pycode}
\begin{pycode}[pe_ss3]
pytex.add_dependencies('/home/chymera/src/faceRT/analysis/signal_detection.py')
pytex.add_dependencies('/home/chymera/src/faceRT/analysis/gen.cfg')
sys.path.append('/home/chymera/src/faceRT/analysis/')
import signal_detection
data_pe_ss3 = signal_detection.main(experiment='6px-4px-6steps', prepixelation=0, source='server', elinewidth=1, ecolor='0.3', make_tight={"pad": 0})
fig_pe_ss3 = latex_figure(save_fig(fig_width=6.64, fig_height=3), caption='Error rates for Hariri-style face matching. Scrambled images for simple visual recognition were preprocessed only with a scrambling cluster of the sizes indicated in the graphic (sizes given in pixels). The error bars represent the standard deviation. Bar-plots with zero-height have been slightly indented for better visibility.', label='r_pe_ss3')
\end{pycode}
\begin{pycode}
pytex.add_dependencies('/home/chymera/src/faceRT/analysis/RTforCategories.py')
pytex.add_dependencies('/home/chymera/src/faceRT/analysis/data_functions.py')
pytex.add_dependencies('/home/chymera/src/faceRT/analysis/gen.cfg')
sys.path.append('/home/chymera/src/faceRT/analysis/')
import RTforCategories
data_pe_cs1 = RTforCategories.main(experiment='11px-4px-5steps', prepixelation=6, source='server', elinewidth=1, ecolor='0.3', make_tight={"pad": 0})
fig_pe_cs1 = latex_figure(save_fig(fig_width=6.64, fig_height=3), caption='Reaction times for Hariri-style face matching. Scrambled images for simple visual recognition were preprocessed with composite cluster and kernel based scrambling, with a constant kernel of \SI{6}{\pixel} and clusters of the sizes indicated in the graphic (sizes given in pixels). Reaction times for non-response trials were counted as \SI{5}{\second}. The error bars represent the standard deviation.', label='r_pe_cs1')
\end{pycode}
\begin{pycode}
pytex.add_dependencies('/home/chymera/src/faceRT/analysis/RTforCategories.py')
pytex.add_dependencies('/home/chymera/src/faceRT/analysis/data_functions.py')
pytex.add_dependencies('/home/chymera/src/faceRT/analysis/gen.cfg')
sys.path.append('/home/chymera/src/faceRT/analysis/')
import RTforCategories
data_pe_cs2 = RTforCategories.main(experiment='6px-4px-5steps', prepixelation=6, source='server', elinewidth=1, ecolor='0.3', make_tight={"pad": 0})
fig_pe_cs2 = latex_figure(save_fig(fig_width=6.64, fig_height=3), caption='Reaction times for Hariri-style face matching. Scrambled images for simple visual recognition were preprocessed with composite cluster and kernel based scrambling, with a constant kernel of \SI{6}{\pixel} and clusters of the sizes indicated in the graphic (sizes given in pixels). Reaction times for non-response trials were counted as \SI{5}{\second}. The error bars represent the standard deviation.', label='r_pe_cs2')
\end{pycode}
\begin{pycode}[pe_cs3]
pytex.add_dependencies('/home/chymera/src/faceRT/analysis/RTforCategories.py')
pytex.add_dependencies('/home/chymera/src/faceRT/analysis/data_functions.py')
pytex.add_dependencies('/home/chymera/src/faceRT/analysis/gen.cfg')
sys.path.append('/home/chymera/src/faceRT/analysis/')
import RTforCategories
data_pe_cs3 = RTforCategories.main(experiment='11px-4px-5steps_narrow-angle', prepixelation=6, source='server', make_std=True, elinewidth=1, ecolor='0.3', make_tight={"pad": 0})
fig_pe_cs3 = latex_figure(save_fig(fig_width=6.64, fig_height=3), caption='Reaction times for Hariri-style face matching. Scrambled images for simple visual recognition were preprocessed with composite cluster and kernel based scrambling, with a constant kernel of \SI{6}{\pixel} and clusters of the sizes indicated in the graphic (sizes given in pixels). Reaction times for non-response trials were counted as \SI{5}{\second}. The light error bars represent the standard deviation, while the dark bars represent the standard error of the mean.', label='r_pe_cs3')
data_pe_cs3['difficulty']=''
data_pe_cs3['scrambled']=''
data_pe_cs3.ix[(data_pe_cs3['CoI'] == 'emotion-hard') | (data_pe_cs3['CoI'] == 'scrambling-11'), 'difficulty'] = 'hard'
data_pe_cs3.ix[(data_pe_cs3['CoI'] == 'emotion-easy') | (data_pe_cs3['CoI'] == 'scrambling-23'), 'difficulty'] = 'easy'
data_pe_cs3.ix[(data_pe_cs3['scrambling'] == 0), 'scrambled'] = 'no'
data_pe_cs3.ix[(data_pe_cs3['scrambling'] != 0), 'scrambled'] = 'yes'
\end{pycode}
\begin{pycode}[pe_cs4]
pytex.add_dependencies('/home/chymera/src/faceRT/analysis/RTdistribution.py')
pytex.add_dependencies('/home/chymera/src/faceRT/analysis/data_functions.py')
pytex.add_dependencies('/home/chymera/src/faceRT/analysis/gen.cfg')
sys.path.append('/home/chymera/src/faceRT/analysis/')
import RTdistribution
keep_scrambling=(0, 11, 23)
data_pe_cs4 = RTdistribution.main(experiment='11px-4px-5steps_narrow-angle', prepixelation=6, keep_scrambling=keep_scrambling, source='server', make_tight={"pad": 0}, print_title=False)
fig_pe_cs4 = latex_figure(save_fig(fig_width=6.64, fig_height=3), caption='Histogram of reaction times for Hariri-style face matching. Scrambled images for simple visual recognition were preprocessed with a scrambling kernel of \SI{6}{\pixel} and with scrambling cluster. Missed targets have not been counted. For this graphic we have only indexed the '+ ', '.join([str(i) for i in keep_scrambling]) +' pixel scrambling cluster trials.', label='r_pe_cs4')
\end{pycode}
\begin{pycode}[pe_cs5]
pytex.add_dependencies('/home/chymera/src/faceRT/analysis/signal_detection.py')
pytex.add_dependencies('/home/chymera/src/faceRT/analysis/data_functions.py')
pytex.add_dependencies('/home/chymera/src/faceRT/analysis/gen.cfg')
sys.path.append('/home/chymera/src/faceRT/analysis/')
import signal_detection
data_pe_cs5 = signal_detection.main(experiment='11px-4px-5steps_narrow-angle', prepixelation=6, source='server', elinewidth=1, ecolor='0.3', make_tight={"pad": 0})
fig_pe_cs5 = latex_figure(save_fig(fig_width=6.64, fig_height=3), caption='Error rates for Hariri-style face matching. Scrambled images for simple visual recognition were preprocessed with composite cluster and kernel based scrambling, with a constant kernel of \SI{6}{\pixel} and clusters of the sizes indicated in the graphic (sizes given in pixels). The error bars represent the standard deviation. Bar-plots with zero-height have been slightly indented for better visibility.', label='r_pe_cs5')
data_pe_cs5['difficulty']=''
data_pe_cs5['scrambled']=''
data_pe_cs5.ix[(data_pe_cs5['CoI'] == 'emotion-hard') | (data_pe_cs5['CoI'] == 'scrambling-11'), 'difficulty'] = 'hard'
data_pe_cs5.ix[(data_pe_cs5['CoI'] == 'emotion-easy') | (data_pe_cs5['CoI'] == 'scrambling-23'), 'difficulty'] = 'easy'
data_pe_cs5.ix[(data_pe_cs5['scrambling'] == 0), 'scrambled'] = 'no'
data_pe_cs5.ix[(data_pe_cs5['scrambling'] != 0), 'scrambled'] = 'yes'
\end{pycode}
\begin{pycode}[p_mr]
pytex.add_dependencies('/home/chymera/src/faceOM/analysis/timecourse_dilation.py')
pytex.add_dependencies('/home/chymera/src/faceRT/analysis/data_functions.py')
pytex.add_dependencies('/home/chymera/src/faceRT/analysis/gen.cfg')
sys.path.append('/home/chymera/src/faceOM/analysis/')
import timecourse_dilation
data_p_mr = timecourse_dilation.corr(source='local', make_tight={"pad": 0})
fig_p_mr = latex_figure(save_fig(fig_width=3.5, fig_height=3.5), caption='Correlation between X-axis and Y-axis pupil diameters on measurement data points. Raw data points are shown in faint gray, and per-participant per-timepoint (relative to trial onset) means are shown in green. The regression line is drawn in magenta. To decrease cluttering, only one in three raw data points is plotted.', label='p_mr')
\end{pycode}
\begin{pycode}[ra_rt]
pytex.add_dependencies('/home/chymera/src/faceOM/analysis/rt.py')
pytex.add_dependencies('/home/chymera/src/faceRT/analysis/data_functions.py')
pytex.add_dependencies('/home/chymera/src/faceRT/analysis/gen.cfg')
sys.path.append('/home/chymera/src/faceOM/analysis/')
import rt
data_ra_rt = rt.main(source='local', make_tight={"pad": 0}, make_std=False, make_scrambled_yn=True, elinewidth=1)
fig_ra_rt = latex_figure(save_fig(fig_width=6.64, fig_height=3), caption='Reaction times for Hariri-style face matching. Reaction times for non-response trials were not counted. The error bars represent the standard deviation.', label='ra_rt')
\end{pycode}
\begin{pycode}[ra_er]
pytex.add_dependencies('/home/chymera/src/faceOM/analysis/er.py')
pytex.add_dependencies('/home/chymera/src/faceRT/analysis/data_functions.py')
pytex.add_dependencies('/home/chymera/src/faceRT/analysis/gen.cfg')
sys.path.append('/home/chymera/src/faceOM/analysis/')
import er
data_ra_er = er.main(source='local', make_tight={"pad": 0}, make_std=False, make_scrambled_yn=True, elinewidth=1)
fig_ra_er = latex_figure(save_fig(fig_width=6.64, fig_height=3), caption='Error rates for Hariri-style face matching. The error bars represent the standard deviation. Bar-plots with zero-height have been slightly indented for better visibility.', label='ra_er')
\end{pycode}
\begin{pycode}[r_pd_ta]
import pandas as pd
pytex.add_dependencies('/home/chymera/src/faceOM/analysis/barplot_dilation.py')
pytex.add_dependencies('/home/chymera/src/faceRT/analysis/data_functions.py')
pytex.add_dependencies('/home/chymera/src/faceRT/analysis/gen.cfg')
sys.path.append('/home/chymera/src/faceOM/analysis/')
import barplot_dilation
data_r_pd_ta = barplot_dilation.main(source='local', make_tight={"pad": 0}, make_std=False, make_sem=True, elinewidth=1)
fig_r_pd_ta = latex_figure(save_fig(fig_width=6.64, fig_height=3.2), caption='Bar plots of pupil dilation integrated over the entire time course. The Y-axis values are proportional to the global fixation (inter-trial interval) pupil size. The error bars represent the standard error of the mean.', label='r_pd_ta')
data_r_pd_ta = data_r_pd_ta["Pupil"].groupby(level=(0,2)).mean()
data_r_pd_ta_av = data_r_pd_ta.reset_index()
data_r_pd_ta_av.columns=["CoI","ID","Pupil"]
\end{pycode}
\begin{pycode}[r_pd]
pytex.add_dependencies('/home/chymera/src/faceOM/analysis/timecourse_dilation.py')
pytex.add_dependencies('/home/chymera/src/faceRT/analysis/data_functions.py')
pytex.add_dependencies('/home/chymera/src/faceRT/analysis/gen.cfg')
sys.path.append('/home/chymera/src/faceOM/analysis/')
import timecourse_dilation
data_r_pd = timecourse_dilation.time_course(source='local', make_tight={"pad": 0}, show=["fix", "all", "rt_all"])
fig_r_pd = latex_figure(save_fig(fig_width=6.64, fig_height=3.2), caption='Pupil dilation over time. The values are proportional to the mean pupil dilation size during fixation (the inter-trial interval). Shaded areas represent the standard errors of the mean. The vertical line represents the mean reaction time. Plotted data was down-sampled to \\SI{15}{\\hertz}.', label='r_pd')
data_r_pd_IDmeans = data_r_pd.set_index(["ID"], append=True, drop=True)
data_r_pd_IDmeans = data_r_pd_IDmeans.groupby(level=(0,2)).mean()
\end{pycode}
\begin{pycode}[r_pd_ed1]
pytex.add_dependencies('/home/chymera/src/faceOM/analysis/timecourse_dilation.py')
pytex.add_dependencies('/home/chymera/src/faceRT/analysis/data_functions.py')
pytex.add_dependencies('/home/chymera/src/faceRT/analysis/gen.cfg')
sys.path.append('/home/chymera/src/faceOM/analysis/')
import timecourse_dilation
data_r_pd_ed1 = timecourse_dilation.time_course(source='local', make_tight={"pad": 0}, show=["easy", "hard", "rt_e", "rt_h"])
fig_r_pd_ed1 = latex_figure(save_fig(fig_width=6.64, fig_height=3.2), caption='Pupil dilation over time for easy and difficult matching. The values are proportional to the mean pupil dilation size during fixation (here 1.0). Shaded areas represent the standard errors of the mean. Vertical lines represent mean reaction times. Plotted data was down-sampled to \\SI{15}{\\hertz}.', label='r_pd_ed1')
data_r_pd_ed1_lm = data_r_pd_ed1.reset_index()
data_r_pd_ed1_lm = data_r_pd_ed1_lm[(data_r_pd_ed1_lm["CoI"] =="easy")|(data_r_pd_ed1_lm["CoI"] =="hard")&(data_r_pd_ed1_lm["measurement"] >= 120)]
\end{pycode}
\begin{pycode}[r_pd_ed2]
pytex.add_dependencies('/home/chymera/src/faceOM/analysis/timecourse_dilation.py')
pytex.add_dependencies('/home/chymera/src/faceRT/analysis/data_functions.py')
pytex.add_dependencies('/home/chymera/src/faceRT/analysis/gen.cfg')
sys.path.append('/home/chymera/src/faceOM/analysis/')
import timecourse_dilation
data_r_pd_ed2 = timecourse_dilation.discrete_time(make_tight={"pad": 0}, show=["easy", "hard"], sample_size = 30)
fig_r_pd_ed2 = latex_figure(save_fig(fig_width=6.64, fig_height=3.2), caption='Pupil dilation over time for easy and difficult matching. The values are proportional to the mean pupil dilation size during fixation (here 1.0). Shaded areas represent the standard errors of the mean. Vertical lines represent mean reaction times. Plotted data was down-sampled to \\SI{15}{\\hertz}.', label='r_pd_ed2')
data_r_pd_ed2 = data_r_pd_ed2.reset_index()
data_r_pd_ed2 = data_r_pd_ed2[(data_r_pd_ed2["CoI"] == "easy") | (data_r_pd_ed2["CoI"] == "hard")]
\end{pycode}
\begin{pycode}[r_pd_hf1]
pytex.add_dependencies('/home/chymera/src/faceOM/analysis/timecourse_dilation.py')
pytex.add_dependencies('/home/chymera/src/faceRT/analysis/data_functions.py')
pytex.add_dependencies('/home/chymera/src/faceRT/analysis/gen.cfg')
sys.path.append('/home/chymera/src/faceOM/analysis/')
import timecourse_dilation
data_r_pd_hf1 = timecourse_dilation.time_course(source='local', make_tight={"pad": 0}, show=["happy", "fearful", "rt_ha", "rt_fe"])
fig_r_pd_hf1 = latex_figure(save_fig(fig_width=6.64, fig_height=3.2), caption='Pupil dilation over time for emotional faces matching. The values are proportional to the mean pupil dilation size during fixation (here 1.0). Shaded areas represent the standard errors of the mean. Vertical lines represent mean reaction times. Plotted data was down-sampled to \\SI{15}{\\hertz}.', label='r_pd_hf1')
data_r_pd_hf1_IDmeans = data_r_pd_hf1.set_index(["ID"], append=True, drop=True)
data_r_pd_hf1_IDmeans = data_r_pd_hf1_IDmeans.groupby(level=(0,1)).mean()
\end{pycode}
\begin{pycode}[r_pd_hf2]
pytex.add_dependencies('/home/chymera/src/faceOM/analysis/timecourse_dilation.py')
pytex.add_dependencies('/home/chymera/src/faceRT/analysis/data_functions.py')
pytex.add_dependencies('/home/chymera/src/faceRT/analysis/gen.cfg')
sys.path.append('/home/chymera/src/faceOM/analysis/')
import timecourse_dilation
data_r_pd_hf2 = timecourse_dilation.discrete_time(make_tight={"pad": 0}, show=["happy", "fearful"], sample_size = 30)
fig_r_pd_hf2 = latex_figure(save_fig(fig_width=6.64, fig_height=3.2), caption='Pupil dilation over time for easy and difficult matching. The values are proportional to the mean pupil dilation size during fixation (here 1.0). Shaded areas represent the standard errors of the mean. Vertical lines represent mean reaction times. Plotted data was down-sampled to \\SI{15}{\\hertz}.', label='r_pd_hf2')
data_r_pd_hf2 = data_r_pd_hf2.reset_index()
data_r_pd_hf2 = data_r_pd_hf2[(data_r_pd_hf2["CoI"] == "happy") | (data_r_pd_hf2["CoI"] == "fearful")]
\end{pycode}
\begin{pycode}[r_pd_es1]
pytex.add_dependencies('/home/chymera/src/faceOM/analysis/timecourse_dilation.py')
pytex.add_dependencies('/home/chymera/src/faceRT/analysis/data_functions.py')
pytex.add_dependencies('/home/chymera/src/faceRT/analysis/gen.cfg')
sys.path.append('/home/chymera/src/faceOM/analysis/')
import timecourse_dilation
data_r_pd_es1 = timecourse_dilation.time_course(source='local', make_tight={"pad": 0}, show=["emotion", "scrambled", "rt_em", "rt_sc"])
fig_r_pd_es1 = latex_figure(save_fig(fig_width=6.64, fig_height=3.2), caption='Pupil dilation over time for matching of either emotional or scrambled faces. The values are proportional to the mean pupil dilation size during fixation (here 1.0). Shaded areas represent the standard errors of the mean. Vertical lines represent mean reaction times. Plotted data was down-sampled to \\SI{15}{\\hertz}.', label='r_pd_es1')
data_r_pd_es1 = data_r_pd_es1.reset_index()
data_r_pd_es1[(data_r_pd_es1["CoI"] =="fearful")]["CoI"] = (data_r_pd_es1[(data_r_pd_es1["CoI"] =="fearful")]["CoI"]+data_r_pd_es1[(data_r_pd_es1["CoI"] =="happy")]["CoI"])/2
data_r_pd_es1.ix[(data_r_pd_es1["CoI"]=="fearful"), "CoI"] = "emotion"
data_r_pd_es1_lm = data_r_pd_es1[(data_r_pd_es1["CoI"] =="scrambled")|(data_r_pd_es1["CoI"] =="emotion")&(data_r_pd_es1["measurement"] >= 120)]
\end{pycode}
\begin{pycode}[r_pd_es2]
pytex.add_dependencies('/home/chymera/src/faceOM/analysis/timecourse_dilation.py')
pytex.add_dependencies('/home/chymera/src/faceRT/analysis/data_functions.py')
pytex.add_dependencies('/home/chymera/src/faceRT/analysis/gen.cfg')
sys.path.append('/home/chymera/src/faceOM/analysis/')
import timecourse_dilation
data_r_pd_es2 = timecourse_dilation.discrete_time(make_tight={"pad": 0}, show=["happy", "fearful"], sample_size = 30)
data_r_pd_es2 = data_r_pd_es2.reset_index()
data_r_pd_es2[(data_r_pd_es2["CoI"] =="fearful")]["CoI"] = (data_r_pd_es2[(data_r_pd_es2["CoI"] =="fearful")]["CoI"]+data_r_pd_es2[(data_r_pd_es2["CoI"] =="happy")]["CoI"])/2
data_r_pd_es2.ix[(data_r_pd_es2["CoI"]=="fearful"), "CoI"] = "emotion"
data_r_pd_es2 = data_r_pd_es2[(data_r_pd_es2["CoI"] == "emotion")|(data_r_pd_es2["CoI"] == "scrambled")]
\end{pycode}
%END FIGURES AND DATA

