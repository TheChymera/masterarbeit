\chapter{Background}
    \section{Pupillometry}\label{sec:b_p}
	Pupillometry refers to the long-standing physiological method of measuring pupil diameter.
	Sporadic studies involving pupil diameter measurements have been performed as early as the beginning of the XIX. century \citep{Loewenfeld1958},
	with the establishment of a scientific community around pupillometry following about oone and a half centuries later.
	The early emergence of pupillometric research owes to the innovation of infra-red imaging to study the eye \citep{Dubois1955}, around the middle of the XX. century (originally for ophthalmologic purposes, where pupillometry also continues to be used \citep{Thompson2012}).
	This development made it possible to measure pupil kinetics without interference from the measurement equipment itself - and remains a standard for even the newest devices of our day \citep{Bradley2010}.
	
	Pupillometry has been in documented use in psychology since 1960, when Hess and Polt published what was arguably the seminal paper of psychophysiological pupillometry in the journal \textit{Science} \citep{HESS1960}.
	In this publication they reported presenting images of partly nude adults and babies to a small sample of participants, and observing increased pupil dilation in males to stimuli depicting women, and increased pupil dilation in women to stimuli depicting either males or babies.
	In light of this manifold interpretable finding, research over the next decade has focused on making the case that pupil dilation is a marker of attraction, or well-being.
	
	Many early publications present varying concepts related to the initial publication by Hess and Polt, and report a robust modulation of pupil diameter by sexual attraction \citep{Goldwater1972, HESS1965} or positive affect \citep{Nunally1967, Bradshaw1967}.
	Even some of these early papers, however, hinted at a broader phenomenon underlying neuropsychological control of pupil size.
	This broader phenomenon was initially termed “general activation” \citep{Nunally1967}, and has over the years been more accurately identified as “cognitive load” \citep{Laeng2011,Zekveld2011}, “attention” \citep{Wykowska2013,Kraemer2000}, or - most recently - “control state” \citep{Hayes2013}.
	Of late, researchers have tentatively tried to establish pupillometry as an indicator of neuronal activity in the Locus Coeruleus (LC; more on this in section~\ref{sec:b_lc}) \citep{Gilzenrat2010}.
	In the research herein presented, we have sought to test the viability of pupillometry as an assessment not of a neuropsychological phenomenon, but of a neurological phenomenon (LC activation) - 
	complementing mostly behavioural previous approaches \citep{Gilzenrat2010,Granholm2004} with concomitant brain imaging.
	
    \section{Pupil Dilation Response}
	Pupilary dilation (mydriasis) is most prominently known as a physiological reaction to low light conditions \citep{Ellis1981}. 
	However, a broad range of mental states - 
	such as alertness \citep{Yoss1970}, arousal \citep{Bradshaw1967}, fatigue \citep{Morad2000}, and surprize \citep{Preuschoff2011} -
	have also been shown to modulate pupil dilation.
	Additionally, application of numerous substances (topically and/or systemically) is known to affect pupil dilation \citep{Theofilopoulos1988, Bye1979, Phillips2000}; to the extent that the method is used to assess drug effects in laboratory animals \citep{Murray1981}.
	Disregulation in tonic pupil dilation is well known in the medical community as a presenting sign for various neurological conditions \citep{Caglayan2013}.
	The phenomenon has, however, also been implicated in non-neuropathic diseases - most notably in autism spectrum disorders \citep{Anderson2013}.
	
	The direct control mechanism of pupil dilation (or contraction for that matter) consists of two smooth muscle sets located in the iris: the iris dilator muscle and the iris sphincter muscle.
	\begin{itemize}
	    \item The iris dilator is a radial muscle located anterior of the pigment epithelium of the iris, and chiefly responsible for iris dilation.
	    This muscle is innervated by the sympathetic nervous system and is controlled via noradrenergic stimulation to its $\alpha_1$-adrenergic receptors \citep{vanAlphen1976}.
	    \item The iris sphincter muscle is located anterior of the iris dilator muscle and chiefly responsible for iris contraction.
	    Its innervation stems from the parasympathetic system and is mainly mediated by M$\mathsf{_3}$-muscarinic acetylcholine receptors \citep{Woldemussie1993,Taylor1974}.
	\end{itemize}
	It should be noted that counterbalanced neurotransmitter effects (such as direct cholinergic inhibition of the iris dilator muscle) have also been documented \citep{Yoshitomi1985}.
	
	In light of these characteristics it is not surprising that topical application of drugs can affect pupil dilation.
	This phenomenon is especially robust and well-documented for cholinergic agonists \citep{Smith1978} and antagonists \citep{Gambill1967} -
	to the extent that pupil dilation is used as an assay for in vivo anticholinergic effects \citep{Bye1979}.
	Serotonin (5-HT) receptors have also been shown to exert an influence on pupil dilation - which has been demonstrated either by topically applied serotonin \citep{KOELLA1962}, or by systemically applied 5-HT receptor agonists \citep{Yu2004}.
	Systemic application of 5-HT receptor agonists (in rats) has furthermore been shown to elicit mydriasis not via peripheral 5-HT receptors, but via CNS 5HT$\mathsf{_{1A}}$ receptors.
	
	Considerable attention was received by the hypothesis that CNS 5-HT (in liaison with the long-standing interest of pupillometrists in affect - detailed under section~\ref{sec:b_p}) may be chiefly implicated in controlling pupil dilation.
	As a most prominent example of pupil-dilation and serotonin interaction, antidepressants of the selective serotonin re-uptake inhibitor (SSRI) class are widely documented to cause tonic mydriasis \citep{Fitzgerald2013,Klein-Schwartz2012}.
	
	Relying on this established phenomenon, at least one study has tried to investigate the value of pupil dilation as a convenient, semi-quantitative 5-HT biomarker \citep{Noehr-Jensen2009}.
	The study found an escitalopram-correlated reduction of light-evoked pupil constriction, but no overall effect of escitalopram (a drug of the SSRI class which the authors use as a proxy for systemic 5-HT) on pupil dilation.
	The authors conclude that pupillometry can, thus, not be used as a 5-HT biomarker.
	In this context we would, however, like to remark that theirs was a chronic test for tonic dilation.
	In fact, the modulation of the pupillary light reflex found by the authors may even indicate that the outcome of their study would have been quite different, had it employed pupillometry to assay phasic dilation related to acute events (such as for instance in a behavioural task).
	In any case, the results of this study conflict with the often reported dose-dependent pupil dilator activity of SSRIs \citep{Nielsen2010,Fitzgerald2013,Klein-Schwartz2012};
	and need not dismiss all hopes of pupillometry as a serotonin biomarker.
	
	Evidence has mounted considerably faster and more consistently on establishing pupil diameter as a biomarker for norepinephrine (NE).
	In light of this, it may be hypothesized that 5-HT works upstream of NE in controlling pupil dilation; and this may lead to the opaquer state of research concerning the correlation of 5-HT and pupil dilation compared to the correlation of NE and pupil dilation.
	Research into the pupil dilation effects of 3,4-methylenedioxymethamphetamine (MDMA) administered in the presence of varying monoamine reuptake inhibitors and monoamine agonists has - however - shown that 5-HT and NE have both an independent and a synergistic effect on pupil dilation \citep{Hysek2012}.
	This conclusion was drawn from the pupil dilation response changes prompted by MDMA being modulated by the selective NE reuptake inhibitor (SNRI) reboxetine and the 5-HT and NE reuptake inhibitor (SNRI) duloxetine, but not by the $\alpha_2$-adrenergic receptor agonist clonidine.
	Additionally, 5-HT increase - as mediated by SSRI administration - has been documented in both anxiety \citep{Blier2007} and cardiovascular \citep{Barton2007} studies to reduce sympathetic (noradrenergic) activation over time. 
	Though these findings do not exclude synergistic effects, we believe it reasonable and more transparent to first investigate NE and 5-HT mediated pupil dilation separately.
	
	A noradrenergic sensitivity of pupil dilation has been demonstrated in studies with the $\alpha_2$-adrenergic receptor agonist clonidine, and the (more broad spectrum) $\alpha_2$-adrenergic receptor antagonist yohimbine \citep{Phillips2000}. 
	As would be expected from the aforementioned findings (and with the $\alpha_2$-adrenoceptor being mainly an autoreceptor), clonidine lead to a decrease in pupil diameter while yohimbine lead to an increase; co-administration showed that these effects counteract each other.
	
	Strengthening the association between the NE system and pupil dilation, evidence from electrophysiology studies performed in macaques \citep{Rajkowski1994} shows a strong correlation and time-lock between neuronal firing in the main noradrenergic nucleus of the brain (the locus coeruleus) and pupil dilation.
	Researchers have sought to further validate the connection between pupil dilation and LC neuronal activity by testing the correlation between pupil response and phemonena which are considered markers of LC activity.
	
	One accessible avenue of doing so is pupil-dilation to LC matching via behavioural patterns \citep{Gilzenrat2010}.
	In the referenced study authors observe that tonic pupil dilation correlates with exploratory behaviour and task disengagement, while phasic pupil dilation correlates with increased task engagement; they argue that this closely mimics documented \citep{Aston-Jones2005} changes in behaviour upon tonic or phasic LC activity.
	In the review of this study it is very important to note that pupil dilation correlates \textit{not} with task difficulty per se, but with task engagement - meaning that if a task is to difficult to fulfil, pupil diameter may not see a phasic increase.
	Interestingly, similar pupil dilation kinetics have been found during off-line thought \citep{Smallwood2011}, and have been again put into liaison with noradrenergic LC activity (as this is known to play a pivotal role in attention control - more under section~\ref{sec:b_lc}).   
	
	Other researchers have tried to test the correlation between physiological phenomena which are strongly suspected of being LC-mediated, and make the case that good correlation between these strengthens the probability of a common cause (the only candidate for which would be LC activation) \citep{Murphy2011}.
	In the annotated example the authors test pupil dilation kinetics against the P3 (also P300) event-related potential (known for its role in decision-making \citep{CHAPMAN1964} and attention \citep{Picton19992}) and observe a strong and positive correlation.
	
	Tests using more direct assessments of LC activation in humans are scarce, traditionally due to methodological restrictions.
	The advent of functional magnetic resonance imaging (fMRI) has also had less impact here than in research on other brain areas due to the nonstandard blood-oxygen level dependent (BOLD) response patterns of brainstem structures and the small size of the nucleus \citep{Astafiev2010}.
	
    \section{Locus Coeruleus}\label{sec:b_lc}
	The Locus Coeruleus (LC) is a bilateral brainstem nucleus located in the pons, close to the superior cerebellar pedunculi.
	Its name and original identification base on the (very faintly) blue appearance of the nucleus in fresh brain tissue \citep{Maeda2000}.
	The distinctive blue colour is attributed to monoamine (in this case noradrenaline) polymerization and is analogous to the colouring of other brainstem nuclei such as the substantia nigra \citep{Mai2011}.
	The right nucleus of the LC has been documented to comprise approx. 33.600 catecholaminergic cells, while the cell count of the left nucleus lies at about 39.100 \citep{Mouton1994}.
	While post mortem dissection-based coordinates for the LC have been documented for centuries \citep{Maeda2000},
	in vivo magnetic-resonance determined coordinates have only recently emerged \citep{Keren2009}. 
	We believe that in light of prospective fMRI study, these data serve as a far better guideline. 
		
	LC activity has been originally identified in response to various painful stimuli, has gradually been expanded to include first non-painful threat signals \citep{Grant1984}, and subsequently many more events.
	The LC is known chiefly for its role in noradrenergic neuronal transmission \citep{Benarroch2009}, and based on this has been implicated in a wide range of phenomena, exceeding but including:
	\begin{multicols}{2}
	    \begin{itemize}
		\item Anxiety \citep{Weiss1994}
		\item Attention \citep{Benarroch2009}
		\item Autism \citep{Mehler2009}
		\item (Emotional) Arousal \citep{Bangasser2011,Benarroch2009}
		\item Depression \citep{Bangasser2011, Weiss1994}
		\item Drug Use \citep{Samuels2008}
		\item Neural plasticity \citep{Benarroch2009}
		\item Neurodegenerative Symptoms \citep{Samuels2008,Gesi2000}
		\item Psychological Stress \citep{Bangasser2011,Benarroch2009}
		\item Posture and Balance \citep{Benarroch2009}
	    \end{itemize}
	\end{multicols}
	
	From the point of view of pupillometry-based neuronal activity assay, the LC is of interest due to the documented correlation and time-lock between pupil dilation and LC neuron activity in macaques \citep{Rajkowski1994}.
	Further electrophysiological findings in macaques coming from the same authors \citep{Rajkowski2004} found a strong correlaton and time-lock between reaction times in behavioural trials and phasic firing of single neurons in the LC.
	Homogeneous results all over the LC additionally prompted the authors to conclude that - at least in view of this paradigm - LC neurons are physiologically similar across the nucleus \citep{Rajkowski2004}.
	
	The noradrenergic activity of the LC is chiefly implicated in attention control \citep{Aston-Jones1994,Gabay2011}, with pupil dilation also mimicking the inferred LC on-line activity in man \citep{Gabay2011}.
	A well-documented model for LC noradrenergic control of attention has emerged under the name of “adaptive gain”.
	This model proposes that noradrenergic neurons of the LC serve to increase activity contrast between certain cortical units to afford more downstream effect to some and less to others; this contrast increase is formally termed “gain”  \citep{Aston-Jones2005}.
	Evidence for this model has emerged from behavioural task engagement and disengagement, evaluated in light of LC neuronal efferences to the orbitofrontal and anterior cingulate cortices (OFC and AFC respectively) \citep{Aston-Jones2005}; both of these structures being well-known for their role in behavioural control \citep{Baxter2013,Kerns2004}.
	The adaptive gain model has also found supporting evidence in humans, where global fluctuations in functional connectivity have been shown to correlate with task difficulty and task-effected pupil dilation \citep{Eldar2013}.
	
	Pharmacological studies in humans have shown that the effects of the wakefulness-promoting drug \citep{Engber1998} modafinil can be reversed by clonidine, presumably due to an LC mediated mechanism of action for modafinil \citep{Hou2005}.
	A high-profile fMRI study published in the journal \textit{Science} reiterated this hypothesis and identified an LC activity shift upon modafinil administration via functional imaging.
	The accuracy of the imaging has, however, subsequently been drawn into question, with commentators \citep{Astafiev2010} pointing at discrepancies between the LC and the brain stem activation coordinates reported.  
	BOLD LC signals have been described as difficult to interpret when using response magnitudes as the main feature, and authors recommended a more time-course sensitive signal evaluation method \citep{Astafiev2010}.
	Though no such recommendation was made, we believe applying pupil diameter as a regressor in a linear model specifying blood-oxygen level dependent (BOLD) response would well complement previous studies. 

	Summarizing the current state of research, we believe that due to the wide-ranging processes which the LC is causally involved in controlling, its study is relevant for a broad number of neuropsychology subfields.
	Especially in liaison with the purported possibility of approximating LC activation via pupillometry \citep{Gilzenrat2010,Murphy2011}; further research could be conducive to better techniques for monitoring and diagnosing affect, attention, and anxiety disorders, as well as gauging or predicting the effectiveness of drugs or therapy.
	Due to the portability \citep{Bradley2010} and (comparatively) reduced costs of pupillometric equipment, we believe the well-informed use of pupillometry in neuropsychology would be of great benefit to personalized treatment and diagnosis.
	We also envision a utility herefor in the study of self-administered (recreational or therapeutic) drugs in natural settings - a research field where equipment portability is of utmost importance.
    \section{Focus}
	In our present study we shall explore:
	\begin{itemize}
	    \item The viability of a novel feature-obfuscation algorithm as a baseline for established visual matching tasks.
	    \item The viability of pupillometry study in the current optometry-fMRI set-up at the Central Institute of Mental Health.
	    \item The time course of pupillometric data as modulated by emotion recognition.
	    \item The time course of pupillometric data as modulated by task difficulty.
	    \item The covariance of Locus Coeruleus activation and pupil dilation during visual matching.
	\end{itemize}
	
