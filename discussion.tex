\chapter{Discussion}
    \section{Preliminary Experiments}\label{sec:d_pe}
	Most conclusions drawn from our preliminary experiments (for instance in section~\ref{sec:m_pe_cs} or section~\ref{sec:r_pe_cs}) are only marginally reliable due to the high inter-run variability noticed throughout our re-runs in section~\ref{sec:r_pe}.
	We believe the chief reason for this is our reduced participant sample (n = 6 to 7), and we would recommend a more thorough examination on a larger population to better support any claims pertaining to accurate reaction time and error rate comparisons.
	
	Some hallmarks, however, have been constant over all our runs, and we believe these deserve to be reported with more pronounced certainty.
	\subsection{Reaction Times over Scrambling Cluster Sizes}
	    Our preliminary experiments show that even when confronted with very high variability in results there is a robust decay of reaction times as scrambling cluster sizes increase.
	    We report this is happening both with composite scrambling and with simple cluster-based scrambling.
	    
	    Though this effect was assumed by us a priori, and is hardly surprising, we hold that it validates our scrambling software for ample uses in Neuropsychology.
    \section{Pyxrand}
    A number of studies\citep{Rakover2013} have used scrambled face stimuli for their experiments.
    The stimuli are often prepared either by hand, or through time-consuming per-picture editing in an image manipulation program.
    Though automated scrambling software already exists \citep{Conway2008}, its scope is a lot narrower and its algorithms are nor released publicly.
    
    In findings 
    
    We recommend our software over possible alternatives due to its time-efficient batch functionality and its capacity to scramble both with and without specifically distorting facial features.
